% LaTeXar con :
%  pdflatex notebook.tex
% o bien,
%  latex notebook.dvi
%  dvipdfm notebook.dvi
%
%	" The PDF file may contain up to 25 pages of reference material, single-sided,
%   letter or A4 size, with text and illustrations readable by a person with
%   correctable eyesight without magnification from a distance of 1/2 meter. "
%
\documentclass[10pt,landscape,twocolumn,a4paper,notitlepage]{article}
\usepackage{hyperref}
\usepackage[spanish, activeacute]{babel}
\usepackage{fancyhdr}
\usepackage{lastpage}
\usepackage{listings}
\usepackage{amssymb}
\usepackage{svnkw}
\usepackage{rotating}

\svnidlong{$HeadURL: http://tpsh.exp.dc.uba.ar/svn/adeymo-acm/notebook/notebook.tex $}{$LastChangedDate: 2011-08-13 01:12:48 -0300 (sáb, 13 ago 2011) $}{$LastChangedRevision: 59 $}{$LastChangedBy: elsantodel90 $}
\def\revision{\svnrev}

%%% M�rgenes
\setlength{\columnsep}{0.25in}    % default=10pt
\setlength{\columnseprule}{0.5pt}    % default=0pt (no line)

\addtolength{\textheight}{2.35in}
\addtolength{\topmargin}{-0.9in}     % ~ -0.5 del incremento anterior

\addtolength{\textwidth}{1.1in}
\addtolength{\oddsidemargin}{-0.60in} % -0.5 del incremento anterior
\setlength{\headwidth}{\textwidth}
\addtolength{\headwidth}{0.1in}

\setlength{\headsep}{0.08in}
\setlength{\parskip}{0in}
\setlength{\headheight}{15pt}
\setlength{\parindent}{0mm}

%%% Encabezado y pie de p�gina
\pagestyle{fancy}
\fancyhead[LO]{\leftmark\ -\ \rightmark}
%\fancyhead[C]{\textbf{AJI-UBA}}
\fancyhead[RO]{\textbf{AJI-UBA} - rev \revision\ - P\'agina \thepage\ de \pageref{LastPage}\ \begin{rotate}{270}\hspace{.8em}\underline{\textbf{Universidad de Buenos Aires - FCEN -- AJI}\hspace{4.1in}\ }\end{rotate}}
\fancyfoot{}
\fancyfoot[RO]{\begin{rotate}{270}\hspace{-10em}\textbf{Page \thepage\ of \pageref{LastPage}}\end{rotate}}
\renewcommand{\headrulewidth}{0.4pt}
\renewcommand{\footrulewidth}{0.4pt}
\renewcommand{\footruleskip}{0.2in}

%%% Configuraci�n de Listings
\lstloadlanguages{C++}
\lstnewenvironment{code}
	{%\lstset{	numbers=none, frame=lines, basicstyle=\small\ttfamily, }%
	 \csname lst@SetFirstLabel\endcsname}
	{\csname lst@SaveFirstLabel\endcsname}
\lstset{% general command to set parameter(s)
	language=C++, basicstyle=\small\ttfamily, keywordstyle=\slshape,
	emph=[1]{tipo,usa}, emphstyle={[1]\sffamily\bfseries},
	morekeywords={tint,forn,forsn},
	basewidth={0.47em,0.40em},
	columns=fixed, fontadjust, resetmargins, xrightmargin=5pt, xleftmargin=15pt,
	flexiblecolumns=false, tabsize=2, breaklines,	breakatwhitespace=false, extendedchars=true,
	numbers=left, numberstyle=\tiny, stepnumber=1, numbersep=9pt,
	frame=l, framesep=3pt,
}

%%% Macros
\def\nbtitle#1{\begin{Large}\begin{center}\textbf{#1}\end{center}\end{Large}}
\def\nbsection#1{\section{#1}}
\def\nbsubsection#1{\subsection{#1}}
\def\nbcoment#1{\begin{small}\textbf{#1}\end{small}}

\newcommand{\comb}[2]{\left( \begin{array}{c} #1 \\ #2 \end{array}\right)}

\begin{document}

\tableofcontents\newpage

%%% El texto propiamente dicho
\nbtitle{AJI-UBA - Reference}
\section{Algoritmos}
\textbf{\#include $<$algorithm$>$ \#include $<$numeric$>$ \\}
\begin{tabular}{|l|l|l|} \hline
\textbf{Algo} & \textbf{Params} &  \textbf{Funcion} \\  \hline
%swap & e1, e2 &  da vuelta e1,e2 & $1$ \\ \hline
sort, stable\_sort & f, l &  ordena el intervalo \\  \hline
partial\_sort & f, m, l, [cmp] & [f,m) son los m-f menores en orden, \\ && [m,l) es el resto en algun orden \\ \hline
%is\_sorted & f, l &  \textit{bool} si esta ordenado \\  \hline
nth\_element & f, nth, l & \textit{void} ordena el n-esimo, y \\ && particiona el resto \\  \hline
fill, fill\_n & f, l / n, elem & \textit{void} llena [f, l) o [f, \\ && f+n) con elem \\  \hline
lower\_bound, upper\_bound & f, l, elem & \textit{it} al primer / ultimo donde se \\ && puede insertar elem para que\\ && quede ordenada \\  \hline
binary\_search & f, l, elem & \textit{bool} esta elem en [f, l) \\  \hline
copy & f, l, resul & hace resul+$i$=f+$i$ $\forall i$ \\  \hline
find, find\_if, find\_first\_of & f, l, elem & \textit{it} encuentra i $\in$[f,l) tq. i$=$elem, \\ & / pred / f2, l2 & pred(i), i$\in$[f2,l2) \\ \hline
count, count\_if & f, l, elem/pred & cuenta elem, pred(i) \\ \hline
search & f, l, f2, l2 & busca [f2,l2) $\in$ [f,l) \\ \hline
replace, replace\_if & f, l, old & cambia old / pred(i) por new \\ & / pred, new & \\ \hline
reverse & f, l & da vuelta \\ \hline
partition, stable\_partition & f, l, pred & pred(i) ad, !pred(i) atras \\ \hline
%min, max & e1, e2 & men / may & $1$ \\ \hline
min\_element, max\_element & f, l, [comp] & \textit{it} min, max de [f,l] \\ \hline
lexicographical\_compare & f1,l1,f2,l2 & \textit{bool} con [f1,l1]<[f2,l2] \\ \hline
next/prev\_permutation & f,l & deja en [f,l) la perm sig, ant \\ \hline
set\_intersection, & f1, l1, f2, l2, res & [res, $\ldots$) la op. de conj\\
set\_difference, set\_union, & & \\
set\_symmetric\_difference, & & \\ \hline
push\_heap, pop\_heap, & f, l, e / e / & mete/saca e en heap [f,l), \\
make\_heap & & hace un heap de [f,l) \\ \hline
is\_heap & f,l & \textit{bool} es [f,l) un heap \\ \hline
accumulate & f,l,i,[op] & \textit{T} $=$ $\sum$/oper de [f,l) \\ \hline
inner\_product & f1, l1, f2, i & \textit{T} $=$ i $+$ [f1, l1) . [f2, $\ldots$ ) \\ \hline
partial\_sum & f, l, r, [op] & r+i = $\sum$/oper de [f,f+i] $\forall i \in$[f,l) \\ \hline
adjacent\_difference & f, l, r, [op] & r[0]=f[0], r[i]=f[i] - f[i-1] $\forall i \in$[1,l-f) \\ \hline
%power & e, i, op & \textit{T} = $e^{n}$\\ \hline
\end{tabular}

\section{Estructuras}
\subsection{Range Minimum Query $\left<n \log n, 1\right>$ (get)}
\nbcoment{Resrtricci'on: $n < 2^{\mathrm{LVL}}$; mn(i, j) incluye i y \textbf{no} incluye j; mn\_init O(n log n)}
\begin{code}
usa: tipo
#define LVL 10
tipo vec[LVL][1<<LVL];
tipo mn(int i, int j) { // intervalo [i,j)
	int p = 31-__builtin_clz(j-i);
	return min(vec[p][i],vec[p][j-(1<<p)]);
}
void mn_init(int n) {
	int mp = 31-__builtin_clz(n);
	forn(p, mp) forn(x, n-(1<<p)) vec[p+1][x] = min(vec[p][x], vec[p][x+(1<<p)]);
}
\end{code}
\subsection{Range Minimum Query $\left<n, \log n\right>$ (get y set)}
\nbcoment{Uso: \texttt{MAXN} es la cantidad m'axima de elementos que se banca la estructura. pget(i, j) incluye i y \textbf{no} incluye j. init(n) O(n). Funciona con cualquier operador ``+'' asociativo y con elemento neutro ``0'' Se inicializa as'i:}
\verb!cin >> n; tipo* v = rmq.init(n); forn(i, n) cin >> v[i]; rmq.updall();!
\begin{code}
#define MAXN 100000
struct rmq {
	int MAX;
	tipo vec[4*MAXN];
	tipo* init(int n) {
		MAX = 1 << (32-__builtin_clz(n)); 
		fill(vec, vec+2*MAX, 0); // 0 = elemento neutro
		return vec+MAX;
	}
	void updall() { dforn(i, MAX) vec[i] = vec[2*i] + vec[2*i+1]; } // + = operacion
	void pset(int i, tipo vl) {
		vec[i+=MAX] = vl;
		while(i) { i /= 2; vec[i] = vec[2*i] + vec[2*i+1]; } // + = operacion
	}
	tipo pget(int i, int j) { return _pget(i+MAX, j+MAX); }
	tipo _pget(int i, int j) {
		tipo res = 0;             // 0 = elemento neutro
		if (j-i <= 0) return res;
		if (i%2) res += vec[i++]; // + = operaci�n
		res += _pget(i/2, j/2);   // + = operaci�n
		if (j%2) res += vec[--j]; // + = operaci�n
		return res;
	}
};
\end{code}
%
\subsection{Cantidad de menores o iguales en O(log n)}
\begin{code}
//insersion y consulta de cuantos <= en log n
struct leqset {
   int maxl; vector<int> c;
   int pref(int n, int l) { return (n>>(maxl-l))|(1<<l); }
   void ini(int ml) { maxl=ml; c=vector<int>(1<<(maxl+1)); }
   //inserta c copias de e, si c es negativo saca c copias
   void insert(int e, int q=1) { forn(l,maxl+1) c[pref(e,l)]+=q; }
   int leq(int e) {
      int r=0,a=1;
      forn(i,maxl) {
         a<<=1; int b=(e>>maxl-i-1)&1;
         if (b) r+=c[a]; a|=b;
      } return r + c[a]; //sin el c[a] da los estrictamente menores
   }
   int size() { return c[1]; }
   int count(int e) { return c[e|(1<<maxl)]; }
};
\end{code}
\subsection{Suffix Array - Longuest Common Prefix}
\begin{code}
typedef unsigned char xchar;
#define MAXN 1000000

int p[MAXN], r[MAXN], t, n;

bool sacmp(int a, int b) { return p[(a+t)%n] < p[(b+t)%n]; }
void bwt(const xchar *s, int nn) {
	n = nn;
	int bc[256];
	memset(bc, 0, sizeof(bc));
	forn(i, n) ++bc[s[i]];
	forn(i, 255) bc[i+1]+=bc[i];
	forn(i, n) r[--bc[s[i]]]=i;
	forn(i, n) p[i]=bc[s[i]];
	
	int lnb,nb = 1;
	for(t = 1; t < n; t*=2) {
		lnb = nb; nb = 0;
		for(int i = 0, j = 1; i < n; i = j++) {
			/*calcular siguiente bucket*/
			while(j < n && p[r[j]] == p[r[i]]) ++j;
			if (j-i > 1) {
				sort(r+i, r+j, sacmp);
				int pk, opk = p[(r[i]+t)%n];
				int q = i, v = i;
				for(; i < j; i++) {
					if (((pk = p[(r[i]+t)%n]) != opk) && !(q <= opk && pk < j)) { opk = pk; v = i; }
					p[r[i]] = v;
				}
			}
			nb++;
		}
		if (lnb == nb) break;
	}
//	prim = p[0];
}

void lcp(const xchar* s, int* h) { /* h could be over r */
	int q = 0, j;
	forn(i,n) if (p[i]) {
		j = r[p[i]-1];
		while(q < n && s[(i+q)%n] == s[(j+q)%n]) ++q;
		h[p[i]-1] = q;
		if (q > 0) --q;
	}
}
\end{code}
\newpage
%----------------------------------------------------------------------%
\section{Geom}\subsection{Point in Poly}\begin{code}
usa: algorithm, vector
// No se porta bien si le preguntas por un punto que esta justo en la frontera del poligono
struct pto { tipo x,y; };
bool pnpoly(vector<pto>&v,pto p){
  unsigned i, j, mi, mj, c = 0;
  for(i=0, j = v.size()-1; i< v.size(); j = i++){
    if((v[i].y<=p.y && p.y<v[j].y) ||
       (v[j].y<=p.y && p.y<v[i].y)){
      mi=i,mj=j; if(v[mi].y>v[mj].y)swap(mi,mj);
      if((p.x-v[mi].x) * (v[mj].y-v[mi].y)
       < (p.y-v[mi].y) * (v[mj].x-v[mi].x)) c^=1;
    }
  } return c;
}
\end{code}\subsection{Convex Hull}\begin{code}
usa: algorithm, vector, sqr
tipo pcruz(tipo x1,tipo y1,tipo x2,tipo y2){return x1*y2-x2*y1;}
struct pto {
  tipo x,y;
  tipo n2(pto &p2)const{
    return sqr(x-p2.x)+sqr(y-p2.y);
  }
} r;
tipo area3(pto a, pto b, pto c){
  return pcruz(b.x-a.x,b.y-a.y,c.x-a.x,c.y-a.y);
}
bool men2(const pto &p1, const pto &p2){
  return (p1.y==p2.y)?(p1.x<p2.x):(p1.y<p2.y);
}
bool operator<(const pto &p1,const pto &p2){
  tipo ar = area3(r,p1,p2);
  return(ar==0)?(p1.n2(r)<p2.n2(r)):ar>0;
  //< clockwise, >counterclockwise
}
typedef vector<pto> VP;
VP chull(VP & l){
  VP res = l;  if(l.size()<3) return res;
  r = *(min_element(res.begin(),res.end(),men2));
  sort(res.begin(),res.end());
  tint i=0;VP ch;ch.push_back(res[i++]);ch.push_back(res[i++]);
  while(i<res.size())  // area3 > clockwise, < counterclockwise
    if(ch.size()>1 && area3(ch[ch.size()-2],ch[ch.size()-1],res[i])<=0)
      ch.pop_back();
    else
      ch.push_back(res[i++]);
  return ch;
}
\end{code}
\subsection{Circulo m\'inimo}
\begin{code}
usa: algorithm, cmath, vector, pto (con < e ==)
usa: sqr, dist2(pto,pto), tint
typedef double tipo;
typedef vector<pto> VP;
struct circ { tipo r; pto c; };
#define eq(a,b) (fabs(a-b)<0.00000000000001)
circ deIni(VP v){ //l.size()<=3
  circ r;  sort(v.begin(), v.end()); unique(v.begin(), v.end());
  switch(v.size()) {
    case 0: r.c.x=r.c.y=0; r.r = -1; break;
    case 1: r.c=v[0]; r.r=0; break;
    case 2: r.c.x=(v[0].x+v[1].x)/2.0;
        r.c.y=(v[0].y+v[1].y)/2.0;
        r.r=dist2(v[0], r.c); break;
    default: {
      tipo A = 2.0 * (v[0].x-v[2].x);tipo B = 2.0 * (v[0].y-v[2].y);
      tipo C = 2.0 * (v[1].x-v[2].x);tipo D = 2.0 * (v[1].y-v[2].y);
      tipo R = sqr(v[0].x)-sqr(v[2].x)+sqr(v[0].y)-sqr(v[2].y);
      tipo P = sqr(v[1].x)-sqr(v[2].x)+sqr(v[1].y)-sqr(v[2].y);
      tipo det = D*A-B*C;
      if(eq(det, 0)) {swap(v[1],v[2]); v.pop_back(); return deIni(v);}
      r.c.x = ( D*R-B*P)/det;
      r.c.y = (-C*R+A*P)/det;
      r.r = dist2(v[0],r.c);
    }
  }
  return r;
}
circ minDisc(VP::iterator ini,VP::iterator fin,VP& pIni){
  VP::iterator ivp;
  int i,cantP=pIni.size();
  for(ivp=ini,i=0;i+cantP<2 && ivp!=fin;ivp++,i++) pIni.push_back(*ivp);
  circ r = deIni(pIni);
  for(;i>0;i--) pIni.pop_back();
  for(;ivp!=fin;ivp++) if (dist2(*ivp, r.c) > r.r){
    pIni.push_back(*ivp);
    if (cantP<2) r=minDisc(ini,ivp,pIni);
    else r=deIni(pIni);
    pIni.pop_back();
  }
  return r;
}
circ minDisc(VP ps){ //ESTA ES LA QUE SE USA
  random_shuffle(ps.begin(),ps.end()); VP e;
  circ r = minDisc(ps.begin(),ps.end(),e);
  r.r=sqrt(r.r); return r;
};
\end{code}
\subsection{M\'aximo rect\'angulo entre puntos}
\begin{code}
usa: vector, map, algorithm
struct pto {
  tint x,y ;bool operator<(const pto&p2)const{
    return (x==p2.x)?(y<p2.y):(x<p2.x);
  }
};
bool us[10005];
vector<pto> v;
tint l,w;
tint maxAr(tint x, tint y,tint i){
  tint marea=0;
  tint arr=0,aba=w;
  bool partido = false;
  for(tint j=i;j<(tint)v.size();j++){
    if(x>=v[j].x)continue;
    tint dx = (v[j].x-x);
    if(!partido){
      tint ar = (aba-arr) * dx;marea>?=ar;
    } else {
      tint ar = (aba-y) * dx;marea>?=ar;
      ar = (y-arr) * dx;marea>?=ar;
    }
    if(v[j].y==y)partido=true;
    if(v[j].y< y)arr>?=v[j].y;
    if(v[j].y> y)aba<?=v[j].y;
  }
  return marea;
}
tint masacre(){
  fill(us,us+10002,false);
  pto c;c.x=0;c.y=0;v.push_back(c);c.x=l;c.y=w;v.push_back(c);
  tint marea = 0;
  sort(v.begin(),v.end());
  for(tint i=0;i<(tint)v.size();i++){
    us[v[i].y]=true;
    marea>?=maxAr(v[i].x,v[i].y,i);
  }
  for(tint i=0;i<10002;i++)if(us[i])marea>?=maxAr(0,i,0);
  return marea;
}
\end{code}
\subsection{M\'axima cantidad de puntos alineados}
\begin{code}
usa: algorithm, vector, map, set, forn, forall(typeof)
struct pto {
  tipo x,y;
  bool operator<(const pto &o)const{
    return (x!=o.x)?(x<o.x):(y<o.y);
  }
};
struct lin{
  tipo a,b,c;//ax+by=c
  bool operator<(const lin& l)const{
    return a!=l.a?a<l.a:(b!=l.b?b<l.b:c<l.c);
  }
};
typedef vector<pto> VP;
tint mcd(tint a, tint b){return (b==0)?a:mcd(b, a%b);}
lin linea(tipo x1, tipo y1, tipo x2, tipo y2){
  lin l;
  tint d = mcd(y2-y1, x1-x2);
  l.a = (y2-y1)/d;
  l.b = (x1-x2)/d;
  l.c = x1*l.a + y1*l.b;
  return l;
}
VP v;
typedef map<lin, int> MLI;
MLI cl;
tint maxLin(){
  cl.clear();
  sort(v.begin(), v.end());
  tint m=1, acc=1;
  forn(i, ((tint)v.size())-1){
    acc=(v[i]<v[i+1])?1:(acc+1);
    m>?=acc;
  }
  forall(i, v){
    set<lin> este;
    forall(j, v){
    if(*i<*j||*j<*i)
      este.insert(linea(i->x, i->y, j->x, j->y));
    }
    forall(l, este)cl[*l]++;
  }
  forall(l, cl){
    m>?= l->second;
  }
  return m;
}
\end{code}
%\newpage
\subsection{Centro de masa y area de un pol\'igono}
\begin{code}
usa: vector, forn
struct pto { tint x,y; };
typedef vector<pto> poly;
tint pcruz(tint x1, tint y1, tint x2, tint y2) { return x1*y2-x2*y1; }
tint area3(const pto& p, const pto& p2, const pto& p3) {
  return pcruz(p2.x-p.x, p2.y-p.y, p3.x-p.x, p3.y-p.y);
}
tint areaPor2(const poly& p) {
  tint a = 0; tint l = p.size()-1;
  forn(i,l-1) a += area3(p[i], p[i+1], p[l]);
  return abs(a);
}
pto bariCentroPor3(const pto& p1, const pto& p2, const pto& p3) {
  pto r;
  r.x = p1.x+p2.x+p3.x; r.y = p1.y+p2.y+p3.y;
  return r;
}
struct ptoD { double x,y; };
ptoD centro(const poly& p) {
  tint a = 0; ptoD r; r.x=r.y=0; tint l = p.size()-1;
  forn(i,l-1) {
    tint act = area3(p[i], p[i+1], p[l]);
    pto pact = bariCentroPor3(p[i], p[i+1], p[l]);
    r.x += act * pact.x; r.y += act * pact.y; a += act;
  } r.x /= (3 * a); r.y /= (3 * a); return r;
}
\end{code}
\subsection{Par de puntos mas cercano}
\begin{code}
usa algorithm, vector, tdbl, tint, tipo, INF, forn, cmath
const tint MAX_N = 10010;
struct pto { tipo x,y;} r;
typedef vector<pto> VP;
#define ord(n,a,b) bool n(const pto &p, const pto &q){ \
  return ((p.a==q.a)?(p.b<q.b):(p.a<q.a));}
#define sqr(a) ((a)*(a))
ord(mx,x,y);
ord(my,y,x);
bool vale(const pto &p){return mx(p,r);};
tipo dist(pto a,pto b){return sqr(a.x-b.x)+sqr(a.y-b.y);}
pto vx[MAX_N];
pto vy[MAX_N];
tint N;
tipo cpair(tint ini, tint fin){
  if(fin-ini==1)return INF;
  if(fin-ini==2)return dist(vx[ini], vx[ini+1]);
  vector<pto> y(fin-ini);
  copy(vy+ini, vy+fin, y.begin());
  tint  m = (ini+fin)/2;
  r = vx[m];
  stable_partition(vy+ini, vy+fin, vale);
  tipo d = min(cpair(ini, m), cpair(m, fin));
  vector<pto> w;
  forn(i, y.size())if(sqr(fabs(y[i].x-vx[m].x))<=d)w.push_back(y[i]);
  forn(i,w.size()){
    for(tint j=i+1;(j<(tint)w.size())
      && sqr(fabs(w[i].y-w[j].y))<d;j++){
      d<?=dist(w[i],w[j]);
    }
  }
  return d;
}
tipo closest_pair(){
  sort(vx, vx+N,mx);
  sort(vy, vy+N,my);
  for(tint i=1;i<N;i++){
    if(vx[i].x==vx[i-1].x && vx[i].y==vx[i-1].y)return 0;
  }
  return sqrt(cpair(0,N));
}
\end{code}
\subsection{CCW}
\begin{code}
struct point {tint x, y;};
int ccw(const point &p0, const point &p1, const point &p2){
    tint dx1, dx2, dy1, dy2;
    dx1 = p1.x - p0.x; dy1 = p1.y - p0.y;
    dx2 = p2.x - p0.x; dy2 = p2.y - p0.y;
    if (dx1*dy2 > dy1*dx2) return +1;
    if (dx1*dy2 < dy1*dx2) return -1;
    if ((dx1*dx2 < 0) || (dy1*dy2 < 0)) return -1;
    if ((dx1*dx1+dy1*dy1) < (dx2*dx2+dy2*dy2))return +1;
    return 0;
}
\end{code}
\subsection{Sweep Line}
\begin{code}
struct pto { tint x,y; bool operator<(const pto&p2)const{
  return (y==p2.y)?(x<p2.x):(y<p2.y);
}};
struct slp{ tint x,y,i;bool f; bool operator<(const slp&p2)const{
  if(y!=p2.y)return y<p2.y;
  if(x!=p2.x)return x<p2.x;
  if(f!=p2.f)return f;
  return i<p2.i;
}};
slp p2slp(pto p,tint i){slp q;q.x=p.x;q.y=p.y;q.i=i;return q;}
tint area3(pto a,pto b,pto c){
  return (b.x-a.x)*(c.y-a.y)-(b.y-a.y)*(c.x-a.x);
}
tint giro(pto a,pto b,pto c){
  tint a3=area3(a,b,c);
  if(a3<0) return -1; if(a3>0)return 1;
  return 0;
}
bool inter(pair<pto,pto> a, pair<pto,pto> b){
  pto p=a.first,q=a.second,r=b.first,s=b.second;
  if(q<p)swap(p,q);if(s<r)swap(r,s);
  if(r<p){swap(p,r);swap(q,s);}
  tint a1=giro(p,q,r),a2=giro(p,q,s);
  if(a1!=0 || a2!=0){
    return (a1!=a2) && (giro(r,s,p)!=giro(r,s,q));
  } else {
    return !(q<r);
  }
}
tint cant_intersec(vector<pair<pto,pto> >&v){
  tint ic=0;
  set<slp> Q; list<tint> T;
  for(tint i=0;i<(tint)v.size();i++){
    slp p1=p2slp(v[i].first,i);slp p2=p2slp(v[i].second,i);
    if(p2<p1)swap(p1,p2);
    p1.f=true;p2.f=false;
    Q.insert(p1);Q.insert(p2);
  }
  while(Q.size()>0){
    slp p = *(Q.begin());Q.erase(p);
    if(p.f){
      for(list<tint>::iterator it=T.begin();it!=T.end();it++)
        if(inter(v[*it],v[p.i]))ic++;
      T.push_back(p.i);
    } else {
      T.erase(find(T.begin(),T.end(),p.i));
    }
  }
  return ic;
}
\end{code}
\subsection{Intersecci'on de segmentos}
\begin{code}
struct pto{tint x,y;};
struct seg{pto f,s;};
tint sgn(tint a){return (a>0LL) - (a<0LL);}
tint pc(pto a, pto b, pto o){return (a.x-o.x)*(b.y-o.y)-(a.y-o.y)*(b.x-o.x);}
tint pe(pto a, pto b, pto o){return (a.x-o.x)*(b.x-o.x)+(a.y-o.y)*(b.y-o.y);}
bool inter(seg a, seg b){
	tint bf = sgn(pc(a.f, a.s, b.f));
	tint bs = sgn(pc(a.f, a.s, b.s));
	tint af = sgn(pc(b.f, b.s, a.f));
	tint as = sgn(pc(b.f, b.s, a.s));
	if(bf*bs<0 && af*as<0) return true; //cruza sin tocar
	if((bf==0 && pe(a.f,a.s,b.f) <= 0) || (bs==0 && pe(a.f,a.s,b.s) <= 0))return true; //b tiene un vertice en a
	if((af==0 && pe(b.f,b.s,a.f) <= 0) || (as==0 && pe(b.f,b.s,a.s) <= 0))return true; //a tiene un vertice en b
	return false;
}
\end{code}
\subsection{Distancia entre segmentos}
\begin{code}
tdbl dist(pto p, seg s){
	tdbl a = fabs(tdbl(pc(s.f, s.s, p)));
	tdbl b = hypot(s.f.x-s.s.x,s.f.y-s.s.y),h=a/b, c = hypot(b, h);
	tdbl d1 = hypot(s.f.x-p.x,s.f.y-p.y), d2 = hypot(s.s.x-p.x,s.s.y-p.y);
	if(b<1e-10 || c <= d1 || c <= d2)return min(d1, d2); else return h;
}
tdbl dist(seg a, seg b){
	return (inter(a, b))?0.0:min(min(dist(a.f, b), dist(a.s, b)), min(dist(b.f, a), dist(b.s, a)));
}
\end{code}
\subsection{Cuentitas}
\begin{code}
usa: cmath, algorithm, tipo
struct pto{tipo x,y;};
struct lin{tipo a,b,c;};
struct circ{pto c; tipo r;};
#define sqr(a)((a)*(a))
const double PI = (2.0 * acos(0.0));
pto punto(tipo x, tipo y){pto r;r.x=x;r.y=y;return r;}
const pto cero = punto(0,0);
pto suma(pto o, pto s, tipo k){
  return punto(o.x + s.x * k, o.y + s.y * k);
}
pto sim(pto p, pto c){return suma(c, suma(p,c,-1), -1);}
pto ptoMedio(pto a, pto b){return punto((a.x+b.x)/2.0,(a.y+b.y)/2.0);}
tipo pc(pto a, pto b, pto o){
  return (b.y-o.y)*(a.x-o.x)-(a.y-o.y)*(b.x-o.x);
}
tipo pe(pto a, pto b, pto o){
  return (b.x-o.x)*(a.x-o.x)+(b.y-o.y)*(a.y-o.y);
}
#define sqrd(a,b) (sqr(a.x-b.x)+sqr(a.y-b.y))
tipo dist(pto a, pto b){return sqrt(sqrd(a,b));}
//#define feq(a,b) (fabs((a)-(b))<0.000000000001) para interseccion
#define feq(a,b) (fabs((a)-(b))<0.000000001)
tipo zero(tipo t){return feq(t,0.0)?0.0:t;}
bool alin(pto a, pto b, pto c){  return feq(0, pc(a,b,c));}
bool perp(pto a1, pto a2, pto b1, pto b2){
  return feq(0, pe(suma(a1, a2, -1.0), suma(b1, b2, -1.0), cero));
}
bool hayEL(tipo A11, tipo A12, tipo A21, tipo A22){
  return !feq(0.0, A22*A11-A12*A21);
}
pto ecLineal(tipo A11, tipo A12, tipo A21, tipo A22, tipo R1, tipo R2){
  tipo det = A22*A11-A12*A21;
  return punto((A22*R1-A12*R2)/det,(A11*R2-A21*R1)/det);
}
lin linea(pto p1, pto p2){
  lin l;
  l.b = p2.x-p1.x;
  l.a = p1.y-p2.y;
  l.c = p1.x*l.a + p1.y*l.b;
  return l;
}
bool estaPL(pto p, lin l){return feq(p.x * l.a + p.y * l.b, l.c);}
bool estaPS(pto p, pto a, pto b){
  return feq(dist(p,a)+dist(p,b),dist(b,a));
}
lin bisec(pto o, pto a, pto b){
  tipo da = dist(a,o);
  return linea(o, suma(a, suma(b,a,-1.0), da / (da+dist(b,o))));
}
bool paral(lin l1, lin l2){return !hayEL(l1.a, l1.b, l2.a, l2.b);}
bool hayILL(lin l1, lin l2){ //!paralelas || misma
  return !paral(l1,l2)|| !hayEL(l1.a, l1.c, l2.a, l2.c);
}
pto interLL(lin l1, lin l2){//li==l2->pincha
  return ecLineal(l1.a, l1.b, l2.a, l2.b, l1.c, l2.c);
}
bool hayILS(lin l, pto b1, pto b2){
  lin b = linea(b1,b2);
  if(!hayILL(l,b))return false;
  if(estaPL(b1,l))return true;
  return estaPS(interLL(l,b), b1,b2);
}
pto interLS(lin l, pto b1, pto b2){
  return interLL(l, linea(b1, b2));
}
pto interSS(pto a1, pto a2, pto b1, pto b2){
  return interLS(linea(a1, a2), b1, b2);
}
bool hayISS(pto a1, pto a2, pto b1, pto b2){
  if (estaPS(a1,b1,b2)||estaPS(a2,b1,b2)) return true;
  if (estaPS(b1,a1,a2)||estaPS(b2,a1,a2)) return true;
  lin a = linea(a1,a2), b = linea(b1, b2);
  if(!hayILL(a,b))return false;
  if(paral(a,b))return false;
  pto i = interLL(a,b);
  //sale(i);sale(a1);sale(a2);sale(b1);sale(b2);cout << endl;
  return estaPS(i,a1, a2) && estaPS(i,b1,b2);
}
tipo distPL(pto p, lin l){
  return fabs((l.a * p.x + l.b * p.y - l.c)/sqrt(sqr(l.a)+sqr(l.b)));
}
tipo distPS(pto p, pto a1, pto a2){
  tipo aa = sqrd(a1, a2);
  tipo d = distPL(p, linea(a1, a2));
  tipo xx = aa+sqr(d);
  tipo a1a1 = sqrd(a1, p);
  tipo a2a2 = sqrd(a2, p);
  if(max(a1a1, a2a2) > xx){
    return sqrt(min(a1a1, a2a2));
  }else{
    return d;
  }
}
//
pto bariCentro(pto a, pto b, pto c){
  return punto(
    (a.x + b.x + c.x) / 3.0,
    (a.y + b.y + c.y) / 3.0);
}
pto circunCentro(pto a, pto b, pto c){
  tipo A = 2.0 * (a.x-c.x);tipo B = 2.0 * (a.y-c.y);
  tipo C = 2.0 * (b.x-c.x);tipo D = 2.0 * (b.y-c.y);
  tipo R = sqr(a.x)-sqr(c.x)+sqr(a.y)-sqr(c.y);
  tipo P = sqr(b.x)-sqr(c.x)+sqr(b.y)-sqr(c.y);
  return ecLineal(A,B,C,D,R,P);
}
pto ortoCentro(pto a, pto b, pto c){
  pto A = sim(a, ptoMedio(b,c));
  pto B = sim(b, ptoMedio(a,c));
  pto C = sim(c, ptoMedio(b,a));
  return circunCentro(A,B,C);
}
pto inCentro(pto a, pto b, pto c){
  return interLL(bisec(a, b, c), bisec(b, a, c));
}
pto rotar(pto p, pto o, tipo s, tipo c){
  //gira cw un angulo de sin=s, cos=c
  return punto(
    o.x + (p.x - o.x) * c + (p.y - o.y) * s,
    o.y + (p.x - o.x) * -s + (p.y - o.y) * c
  );
}
bool hayEcCuad(tipo a, tipo b, tipo c){//a*x*x+b*x+c=0 tiene sol real?
  if(feq(a,0.0))return false;
  return zero((b*b-4.0*a*c)) >= 0.0;
}
pair<tipo, tipo> ecCuad(tipo a, tipo b, tipo c){//a*x*x+b*x+c=0
  tipo dx = sqrt(zero(b*b-4.0*a*c));
  return make_pair((-b + dx)/(2.0*a),(-b - dx)/(2.0*a));
}
bool adentroCC(circ g, circ c){//c adentro de g sin tocar?
  return g.r > dist(g.c, c.c) + c.r || !feq(g.r, dist(g.c, c.c) + c.r);
}
bool hayICL(circ c, lin l){
  if(feq(0,l.b)){
    swap(l.a, l.b);
    swap(c.c.x, c.c.y);
  }
  if(feq(0,l.b))return false;
  return hayEcCuad(
    sqr(l.a)+sqr(l.b),
    2.0*l.a*l.b*c.c.y-2.0*(sqr(l.b)*c.c.x+l.c*l.a),
    sqr(l.b)*(sqr(c.c.x)+sqr(c.c.y)-sqr(c.r))+sqr(l.c)-2.0*l.c*l.b*c.c.y
  );
}
pair<pto, pto> interCL(circ c, lin l){
  bool sw=false;
  if(sw=feq(0,l.b)){
    swap(l.a, l.b);
    swap(c.c.x, c.c.y);
  }
  pair<tipo, tipo> rc = ecCuad(
    sqr(l.a)+sqr(l.b),
    2.0*l.a*l.b*c.c.y-2.0*(sqr(l.b)*c.c.x+l.c*l.a),
    sqr(l.b)*(sqr(c.c.x)+sqr(c.c.y)-sqr(c.r))+sqr(l.c)-2.0*l.c*l.b*c.c.y
  );
  pair<pto, pto> p(
    punto(rc.first, (l.c - l.a * rc.first) / l.b),
    punto(rc.second, (l.c - l.a * rc.second) / l.b)
  );
  if(sw){
    swap(p.first.x, p.first.y);
    swap(p.second.x, p.second.y);
  }
  return p;
}
bool hayICC(circ c1, circ c2){
  lin l;
  l.a = c1.c.x-c2.c.x;
  l.b = c1.c.y-c2.c.y;
  l.c = (sqr(c2.r)-sqr(c1.r)+sqr(c1.c.x)-sqr(c2.c.x)+sqr(c1.c.y)
    -sqr(c2.c.y))/2.0;
  return hayICL(c1, l);

}
pair<pto, pto> interCC(circ c1, circ c2){
  lin l;
  l.a = c1.c.x-c2.c.x;
  l.b = c1.c.y-c2.c.y;
  l.c = (sqr(c2.r)-sqr(c1.r)+sqr(c1.c.x)-sqr(c2.c.x)+sqr(c1.c.y)
    -sqr(c2.c.y))/2.0;
  return interCL(c1, l);
}
\end{code}
\newpage
\section{Grafos}%\subsection{Dijsktra}%
\subsection{Kruskal \& Union-Find}
\begin{code}
usa: vector, utility, forn
typedef pair< tint, pair<int,int> > eje;
int n; vector<eje> ejes; //grafo n=cant nodos
#define MAXN 100000
int _cl[MAXN]; //empieza con todos en -1
int cl(int i) { return (_cl[i] == -1 ? i : _cl[i] = cl(_cl[i])); }
void join(int i, int j) { if(cl(i)!=cl(j)) _cl[cl(i)] = cl(j); }
tint krus() {
  if (n==1) return 0;
  sort(ejes.begin(), ejes.end());
  int u = 0; tint t = 0;
  memset(_cl,-1,sizeof(_cl));
  forn(i,ejes.size()) {
    eje& e = ejes[i];
    if (cl(e.second.first) != cl(e.second.second)) {
      u++; t += e.first; if(u==n-1) return t;
      join(e.second.first, e.second.second);
    }
  } return -1; //-1 es que no es conexo
}
\end{code}
\subsection{Bellman-Ford}
\begin{code}
bool bellmanFord(int n){
  int i,o,d;
  static int dis[2*MAX+2];
  fill(dis,dis+n,INF);
  dis[ORIGEN]=0;
  camino[ORIGEN]=0;
  bool cambios=true;
  for(i=0;i<n && cambios;i++){
    cambios=false;
    forn(o,n)forn(d,n){
      if (dis[d]>dis[o]+ejes[o][d].costo){
        dis[d]=dis[o]+ejes[o][d].costo;
        camino[d]=o;
        cambios=true;
      }
    }
    return dis[DESTINO]<INF;
};
\end{code}
\subsection{Floyd-Warhsall}\begin{code}
tint n;tint mc[MAXN][MAXN]; //grafo (mat de long de ady)
void floyd(){
  forn(k,n)forn(i,n)forn(j,n) mc[i][j] <?= mc[i][k]+mc[k][j];
}
\end{code}\subsection{Edmond-Karp}\begin{code}
usa: map,algorithm,queue
struct Eje{ long f,m; long d(){return m-f;}};
typedef map <int, Eje> MIE; MIE red[MAX_N];
int N,F,D;
void iniG(int n, int f, int d){N=n; F=f; D=d;fill(red, red+N, MIE());}
void aEje(int d, int h, int m){
  red[d][h].m=m;red[d][h].f=red[h][d].m=red[h][d].f=0;
}
#define DIF_F(i,j) (red[i][j].d())
#define DIF_FI(i)  (i->second.d())
int v[MAX_N];
long camAu(){
  fill(v, v+N,-1);
  queue<int> c;
  c.push(F);
  while(!(c.empty()) && v[D]==-1){
    int n = c.front(); c.pop();
    for(MIE::iterator i = red[n].begin(); i!=red[n].end(); i++){
      if(v[i->first]==-1 && DIF_FI(i) > 0){
        v[i->first]=n;
        c.push(i->first);
      }
    }
  }
  if(v[D]==-1)return 0;
  int n = D;
  long f = DIF_F(v[n], n);
  while(n!=F){
    f<?=DIF_F(v[n], n);
    n=v[n];
  }
  n = D;
  while(n!=F){
    red[n][v[n]].f=-(red[v[n]][n].f+=f);
    n=v[n];
  }
  return f;
}
long flujo(){long tot=0, c;do{tot+=(c=camAu());}while(c>0); return tot;}
\end{code}
\subsection{Preflow-push}
\begin{code}
usa: algorithm, list, forn
#define MAX_N 200
typedef list<tint> lint;
typedef lint::iterator lintIt;
//usadas para el flujo
tint f[MAX_N][MAX_N]; //flujo
tint e[MAX_N]; //exceso
tint h[MAX_N]; //altura
lintIt cur[MAX_N];

//esto representa el grafo que hay que armar
lint ady[MAX_N]; //lista de adyacencias (para los dos lados)
tint c[MAX_N][MAX_N]; //capacidad (para los dos lados)
tint n; //cant de nodos

tint cf(tint i, tint j) { return c[i][j] - f[i][j]; }

void push(tint i, tint j) {
  tint p = min(e[i], cf(i,j));
  f[j][i] = -(f[i][j] += p);
  e[i] -= p;
  e[j] += p;
}
void lift(tint i) {
  tint hMin = n*n;
  for(lintIt it = ady[i].begin() ; it != ady[i].end() ; ++it) {
    if (cf(i, *it) > 0) hMin = min(hMin, h[*it]);
  }
  h[i] = hMin + 1;
}
void iniF(tint desde)
{
  forn(i,n) {
    h[i] = e[i] = 0;
    forn(j,n) f[i][j] = 0;
    cur[i] = ady[i].begin();
  }
  h[desde] = n;
  for(lintIt it = ady[desde].begin() ; it != ady[desde].end() ; ++it)
  {
    f[*it][desde] = -(f[desde][*it] = e[*it] = c[desde][*it]);
  }
}
void disch(tint i) {
  while(e[i] > 0)  {
    lintIt& it = cur[i];
    if (it == ady[i].end()) {lift(i); it = ady[i].begin();}
    else if (cf(i,*it) > 0 && h[i] == h[*it] + 1) push(i,*it);
    else ++it;
  }
}
tint calcF(tint desde, tint hasta) {
  iniF(desde);
  lint l;
  forn(i,n) {if (i != desde && i != hasta) l.push_back(i);}
  for(lintIt it = l.begin() ; it != l.end() ; ++it) {
    tint antH = h[*it];
    disch(*it);
    if (h[*it] > antH) { //move to front
      l.push_front(*it);
      l.erase(it);
      it = l.begin();
    }
  } return e[hasta];
}
void addEje(tint a, tint b, tint ca) {
  //requiere reiniciar las capacidades
  if (c[a][b] == 0) {//soporta muchos ejes mismo par de nodos
    ady[a].push_back(b);
    ady[b].push_back(a);
  }
  c[b][a] = c[a][b] += ca;
}
void iniGrafo(tint nn) {  //requiere n ya leido
	n=nn;
  forn(i,n) {
    forn(j,n) c[i][j] = 0;
    //solo si se usa la version de addeje con soporte multieje
    ady[i].clear();
  }
}
\end{code}
\subsection{Flujo de costo m'inimo}
\begin{code}
#define MAXN 100
const int INF = 1<<30;
struct Eje{
	int f, m, p;
	int d(){return m-f;}
};
Eje red[MAXN][MAXN];
int adyc[MAXN], ady[MAXN][MAXN];
int N,F,D;
void iniG(int n, int f, int d){ // n, fuente, destino
	N=n;F=f;D=d;
	fill(red[0], red[N], (Eje){0,0,0});
	fill(adyc, adyc+N, 0);
}
void aEje(int d, int h, int m, int p){
	red[h][d].p = -(red[d][h].p = p);
	red[d][h].m = m; //poner [h][d] en m tambien para hacer eje bidireccional
	ady[d][adyc[d]++]=h; ady[h][adyc[h]++]=d;
}
int md[MAXN],vd[MAXN];
int camAu(int &v){
	fill(vd, vd+N, -1);
	vd[F]=F; md[F]=0;
	forn(rep, N)forn(i, N)if(vd[i]!=-1)forn(jj, adyc[i]){
		int j = ady[i][jj], nd = md[i]+red[i][j].p;
		if(red[i][j].d()>0)if(vd[j]==-1 || md[j] > nd)md[j]=nd,vd[j]=i;
	}
	v=0;
	if(vd[D]==-1)return 0;
	int f = INF;
	for(int n=D;n!=F;n=vd[n]) f <?= red[vd[n]][n].d();
	for(int n=D;n!=F;n=vd[n]){
		red[n][vd[n]].f=-(red[vd[n]][n].f+=f);
		v += red[vd[n]][n].p * f;
	}
	return f;
}
int flujo(int &r){ // r = costo, return = flujo
	r=0; int v,f=0, c;
	while((c = camAu(v)))r += v,f += c;
	return f;
}
\end{code}
%\begin{code}
%usa: algorithm, cassert, vector, set, forn
%#define MAX_N 256
%#define MAX_M 1024
%const int INF = (1<<31)-1;
%using namespace std;
%struct Eje{
%  int d,l; /*destino y longitud*/
%};
%Eje ad[MAX_N][MAX_N]; /*adjacencias*/
%int cAd[MAX_N];      /*cantidad de adjacencias*/
%int velocidades[MAX_M+1];
%void cargarGrafo(int n,int m){
%    int i,a,b,l;
%  fill(cAd,cAd+n,0);
%    forn(i,m){
%      scanf("%d%d%d",&a,&b,&l);
%        a--;b--;
%        ad[a][cAd[a]].d=b;
%        ad[a][cAd[a]++].l=l;
%        ad[b][cAd[b]].d=a;
%        ad[b][cAd[b]++].l=l;
%        velocidades[i]=l;
%    }
%    sort(velocidades,velocidades+m);
%    velocidades[m]=INF;
%}
%/*VECTOR PARA HACER DIJSTRA*/
%int distancia[MAX_N];
%bool noKnow[MAX_N];
%int minMax(int o,int d,int n, int cota){
%  register int i,best,iBest;
%  /*cargo todos los nodos con distancia INF menos el origen*/
%  fill(distancia,distancia+n,INF);
%    fill(noKnow,noKnow+n,true);
%  distancia[o]=0;
%  best=0;
%  while(best<INF){
%      best=INF;
%        forn(i,n){
%          if (noKnow[i] && distancia[i]<best){
%              best=distancia[i];
%                iBest=i;
%            }
%        }
%      if (best<INF){
%      noKnow[iBest]=false;
%      forn(i,cAd[iBest]){ /*recorro todos los adjacentes*/
%            if (ad[iBest][i].l>=cota && distancia[ad[iBest][i].d]>
%	    max(distancia[iBest],ad[iBest][i].l))
%              distancia[ad[iBest][i].d]=
%	      max(distancia[iBest],ad[iBest][i].l);
%            }
%        }
%  }
%  return distancia[d];
%};
%
%int maxMin(int o,int d,int n, int cota){
%  register int i,best,iBest;
%
%    /*cargo todos los nodos con distancia INF menos el origen*/
%  fill(distancia,distancia+n,0);
%    fill(noKnow,noKnow+n,true);
%  distancia[o]=INF;
%
%  best=INF;
%  while(best>0){
%      best=0;
%        forn(i,n){
%          if (noKnow[i] && distancia[i]>best){
%              best=distancia[i];
%                iBest=i;
%            }
%        }
%
%        if (best>0){
%      noKnow[iBest]=false;
%      forn(i,cAd[iBest]){ /*recorro todos los adjacentes*/
%            if (ad[iBest][i].l<=cota && distancia[ad[iBest][i].d]<
%              min(distancia[iBest],ad[iBest][i].l))
%                 distancia[ad[iBest][i].d]=
%                 min(distancia[iBest],ad[iBest][i].l);
%            }
%        }
%  }
%  return distancia[d];
%}
%
%void resolver(int n,int m,int vi,int vf){
%  int cantQ,o,d,best,cotaInf,minV,maxV;
%
%    scanf("%d",&cantQ);
%    while(cantQ--){
%      scanf("%d%d",&o,&d);o--;d--;
%        cotaInf=velocidades[0];
%        best=INF;
%        minV=-1;maxV=INF-1;
%        while(minV!=0 && maxV!=INF && best!=0 ){
%          best=min(best,maxV-minV);
%            minV=maxV;
%            maxV=minMax(o,d,n,cotaInf);
%            if (minV==maxV) continue;
%            minV=maxMin(o,d,n,maxV);
%            cotaInf=*(upper_bound(velocidades,velocidades+m,minV));
%            cotaInf=*(upper_bound(velocidades,velocidades+m,cotaInf));
%        }
%        best+=vi+vf;
%        printf("%d\n",best);
%    };
%};
%\end{code}
\subsection{Matching perfecto de costo m\'aximo - Hungarian O(N\^\ 3)}
\begin{code}
#define MAXN 256
#define INFTO 0x7f7f7f7f
int n;
int mt[MAXN][MAXN]; // Matriz de costos (X * Y)
int xy[MAXN], yx[MAXN]; // Matching resultante (X->Y, Y->X)

int lx[MAXN], ly[MAXN], slk[MAXN], slkx[MAXN], prv[MAXN];
char S[MAXN], T[MAXN];
void updtree(int x) {
	forn(y, n) if (lx[x] + ly[y] - mt[x][y] < slk[y]) {
		slk[y] = lx[x] + ly[y] - mt[x][y];
		slkx[y] = x;
} }
int hungar() {
	forn(i, n) {
		ly[i] = 0;
		lx[i] = *max_element(mt[i], mt[i]+n);
	}
	memset(xy, -1, sizeof(xy));
	memset(yx, -1, sizeof(yx));
	forn(m, n) {
		memset(S, 0, sizeof(S));
		memset(T, 0, sizeof(T));
		memset(prv, -1, sizeof(prv));
		memset(slk, 0x7f, sizeof(slk));
		queue<int> q;
		#define bpone(e, p) { q.push(e); prv[e] = p; S[e] = 1; updtree(e); }
		forn(i, n) if (xy[i] == -1) { bpone(i, -2); break; }
		int x=0, y=-1;
		while (y==-1) {
			while (!q.empty() && y==-1) {
				x = q.front(); q.pop();
				forn(j, n) if (mt[x][j] == lx[x] + ly[j] && !T[j]) {
					if (yx[j] == -1) { y = j; break; }
					T[j] = 1;
					bpone(yx[j], x);
				}
			}
			if (y!=-1) break;
			int dlt = INFTO;
			forn(j, n) if (!T[j]) dlt = min(dlt, slk[j]);
			forn(k, n) {
				if (S[k]) lx[k] -= dlt;
				if (T[k]) ly[k] += dlt;
				if (!T[k]) slk[k] -= dlt;
			}
			forn(j, n) if (!T[j] && !slk[j]) {
				if (yx[j] == -1) {
					x = slkx[j]; y = j; break;
				} else {
					T[j] = 1;
					if (!S[yx[j]]) bpone(yx[j], slkx[j]);
				}
			}
		}
		if (y!=-1) {
			for(int p = x; p != -2; p = prv[p]) {
				yx[y] = p;
				int ty = xy[p]; xy[p] = y; y = ty;
			}
		} else break;
	}
	int res = 0;
	forn(i, n) res += mt[i][xy[i]];
	return res;
}
\end{code}
\subsection{Camino/Circuito Euleriano}
\begin{code}
usa: algorithm, vector, list, forn
typedef string ejeVal;
#define MENORATODOS ""
typedef pair<ejeVal, tint> eje;
tint n; vector<eje> ady[MAXN]; tint g[MAXN];
  //grafo (inG = in grado o grado si es no dir)
tint aux[MAXN];
tint pinta(tint f) {
  if (aux[f]) return 0;
  tint r = 1; aux[f] = 1;
  forn(i,ady[f].size()) r+=pinta(ady[f][i].second);
  return r;
}
tint compCon() { fill(aux, aux+n, 0); tint r=0; forn(i,n) if (!aux[i]) { r++; pinta(r); } return r; }
bool isEuler(bool path, bool dir) {
  if (compCon() > 1) return false; tint c = (path ? 2 : 0);
  forn(i,n) if(!dir ? ady[i].size() % 2 : g[i] != 0) {
    if (dir && abs(g[i]) > 1) return false;
    c--; if(c<0) return false; }
  return true;
}
bool findCycle(tint f, tint t, list<tint>& r) {
  if (aux[f] >= ady[f].size()) return false;
  tint va = ady[f][aux[f]++].second;
  r.push_back(va);
  return (va != t ? findCycle(va, t, r) : true);
}
list<tint> findEuler(bool path) { //always directed, no repeated values
  if (!isEuler(path, true)) return list<tint>();
  bool agrego = false;
  if (path) {
    tint i = max_element(g, g + n)-g;
    tint j = min_element(g, g + n)-g;
    if (g[i] != 0) { ady[i].push_back( eje(MENORATODOS, j) ); agrego = true; }
  }
  tint x = -1;
  forn(i,n) {sort(ady[i].begin(), ady[i].end()); if (x<0 || ady[i][0] < ady[x][0]) x=i;}
  fill(aux, aux+n, 0);
  list<tint> r; findCycle(x,x,r); if (!agrego) r.push_front(r.back());
  list<tint> aux; bool find=false;
  list<tint>::iterator it = r.end();
  do{ if (!find) --it;
    for(find=findCycle(*it, *it, aux);!aux.empty();aux.pop_front()) it = r.insert(++it, aux.front());
  } while (it != r.begin());
  return r;
}
\end{code}
\subsection{Erd\"os-Gallai}
\begin{code}
includes: algorithm, functional, numeric, forn
tint n;tint d[MAXL]; //grafo
tint sd[MAXL]; //auxiliar
bool graphical() {
  if (accumulate(d, d+n, 0) % 2 == 1) return false;
  sort(d, d+n, greater<tint>()); copy(d, d+n, sd);
  forn(i,n) sd[i+1]+=sd[i];
  forn(i,n) {
    if (d[i] < 0) return false;
    tint j = lower_bound(d+i+1, d+n, i+1, greater<tint>()) - d;
    if (sd[i] > i*(i+1) + sd[n-1] - sd[j-1] + (j-i-1)*(i+1))
      return false;
  } return true;
}
\end{code}
\subsection{Puntos de articulaci\'on}
\begin{code}
usa: vector, forn
typedef vector< vector<tint> > adyList;
adyList g; //EL GRAFO
vector<bool> artR; tint artT;
vector<tint> artB,artD;
void dfs(tint ant, tint v) {
  artB[v] = artT; artD[v] = artT++;
  forn(i, g[v].size()) if (artD[g[v][i]] == -1) {
      if (!v && i) artR[0]=true;
      tint w = g[v][i]; dfs(v, w);
      if (artB[w] < artD[v]) artB[v] <?= artB[w];
      else if (artB[w] >= artD[v] && v) artR[v]=true;
  } else if (g[v][i] != ant) {
      artB[v] <?= artD[g[v][i]];
  }
}
vector<bool> artPoints() {
  //dice true en los que son ptos de articulacion
  artR.clear(); artR.insert(artR.begin(), g.size(), false);
  if (!g.empty()) {
      artD.clear(); artD.insert(artD.begin(), g.size(), -1);
      artB.resize(g.size()); artT = 0; dfs(-1, 0);
  }
  return artR;
}
\end{code}
\subsection{Grafo cactus}
\textbf{Def:} Un grafo es cactus \textit{sii} todo eje est'a en a lo sumo un ciclo.
\begin{code}
struct eje { int t,i; };
typedef vector<eje> cycle;
int n,m,us[MAXM],pa[MAXN],epa[MAXN],tr[MAXM];
vector<eje> ady[MAXN];
void iniG(int nn) { n=nn; m=0; fill(ady,ady+n,vector<eje>());
fill(pa,pa+n,-1); }
//f:from t:to d:0 si no es dirigido y 1 si es dirigido
void addE(int f, int t, int d) {
  ady[f].push_back((eje){t,m});
  if (!d) ady[t].push_back((eje){f,m}), tr[m]=0;
  us[m++]=0;
}
//devuelve false si algun eje esta en mas de un ciclo
bool cycles(vector<cycle>& vr,int f=0,int a=-2,int ai=-2) {
  int t; pa[f]=a; epa[f]=ai;
  forn(i,ady[f].size()) if (!tr[ady[f][i].i]++) if (pa[t=ady[f][i].t]!=-1) {
    cycle c(1,ady[f][i]); int ef=f;
    do {
      if (!ef) return 0;
      eje e=ady[pa[ef]][epa[ef]];
      if (us[e.i]++) return 0;
      c.push_back(e);
    } while ((ef=pa[ef])!=t);
    vr.push_back(c);
  } else if (!cycles(vr,t,f,i)) return 0;
  return 1;
};
\end{code}
\newpage
\section{Matem\'atica}
\subsection{Algoritmos de cuentas}
\subsubsection{MCD}\begin{code}
tint mcd(tint a, tint b){ return (a==0)?b:mcd(b%a, a);}
struct dxy {tint d,x,y;};
dxy mcde(tint a, tint b) {
  dxy r, t;
  if (b == 0) {
    r.d = a; r.x = 1; r.y = 0;
  } else {
    t = mcde(b,a%b);
    r.d = t.d; r.x = t.y;
    r.y = t.x - a/b*t.y;
  }
  return r;
}
\end{code}
%
\subsubsection{N'umero combinatorio}\begin{code}
tint _comb[MAXMEM][MAXMEM];
tint comb(tint n, tint m) {
  if (m<0||m>n)return 0;if(m==0||m==n)return 1;
  if (n >= MAXMEM) return comb(n-1,m-1)+comb(n-1,m);
  tint& r = _comb[n][m];
  if (r == -1) r = comb(n-1,m-1)+comb(n-1,m);
  return r;
}
// Bolas en Cajas
tint bolEnCaj(tint b, tint c) {return comb(c+b-1,b); }
\end{code}
%
\subsubsection{Teorema Chino del Resto}\begin{code}
usa: mcde
#define modq(x) (((x)%q+q)%q)
tint tcr(tint* r, tint* m, int n) { // x \equiv r_i (m_i) i \in [0..n)
	tint p=0, q=1;
	forn(i, n) {
		p = modq(p-r[i]);
		dxy w = mcde(m[i], q);
		if (p%w.d) return -1; // sistema incompaible
		q = q / w.d * m[i];
		p = modq(r[i] + m[i] * p / w.d * w.x);
	}
	return p; // x \equiv p (q)
}
\end{code}
%
\subsubsection{Potenciaci'on en O(log(e))}%
\begin{code}
tint potLog(tint b, tint e, tint m) {
    if (!e) return 1LL;
    tint r=potLog(b, e>>1, m);
    r=(r*r)%m;
    return (e&1)?(r*b)%m:r;
}
\end{code}
\subsubsection{Longitud de los n'umeros de 1 a N}%
\begin{code}
tint sumDig(tint n, tint m){ // resultado modulo m
    tint b=10, d=1, r=0;
    while(b<=n){
        r = (r + (b-b/10LL)*(d++))%m;
        b*=10LL;
    }
    return (r + (n-b/10LL+1LL)*d)%m;
}
\end{code}
%
\subsection{Teoremas y propiedades}%
\subsubsection{Ecuaci'on de grafo planar}
$regiones = ejes - nodos + componentesConexas + 1$
\subsubsection{Ternas pitag'oricas}
Hay ternas pitag'oricas de la forma: $(a,b,c) = ( m^2-n^2 , 2\cdot m\cdot n, m^2+n^2 ) \forall m > n > 0 $\\
y son primitivas \emph{sii} $(2 | m\cdot n) \land (mcd(m, n) = 1)$\\
(Todas las primitivas (con $(a,b)$ no ordenado) son de esa forma.) Obs: $(\mathrm{m}+i\mathrm{n})^2 = \mathrm{a}+i\mathrm{b}$
\subsubsection{Teorema de Pick}
$A = I + \frac{B}{2} - 1$, donde $I =$ interior y $B =$ borde
%
\subsubsection{Propiedadas varias}
$\sum_{i=0}^n{r^i} = \frac{r^{n+1}-1}{r-1}$ ; $\sum_{i=1}^n{i^2} = \frac{n\cdot(n+1)\cdot(2n+1)}{6}$ ;
$\sum_{i=1}^n{i^3} = \left(\frac{n\cdot(n+1)}{2}\right)^2$ \\
$\sum_{i=1}^n{i^4} = \frac{n\cdot(n+1)\cdot(2n+1)\cdot(3n^2+3n-1)}{12}$ ;
$\sum_{i=1}^{ n} i^5 = \left(\frac{ n \cdot (n+1)}{2}\right) ^2 \cdot \frac{2 n ^2 + 2n - 1}{3}$ \\
%
$\sum_{i=1}^n{\comb{n-1}{i-1}} = 2^{n-1}$ ; $\sum_{i=1}^n{i\cdot\comb{n-1}{i-1}} = n\cdot2^{n-1}$ \\
%
\subsection{Tablas y cotas}
\subsubsection{Primos}
2 3 5 7 11 13 17 19 23 29
31 37 41 43 47 53 59 61 67 71
73 79 83 89 97 101 103 107 109 113
127 131 137 139 149 151 157 163 167 173
179 181 191 193 197 199 211 223 227 229
233 239 241 251 257 263 269 271 277 281
283 293 307 311 313 317 331 337 347 349
353 359 367 373 379 383 389 397 401 409
419 421 431 433 439 443 449 457 461 463
467 479 487 491 499 503 509 521 523 541
547 557 563 569 571 577 587 593 599 601
607 613 617 619 631 641 643 647 653 659
661 673 677 683 691 701 709 719 727 733
739 743 751 757 761 769 773 787 797 809
811 821 823 827 829 839 853 857 859 863
877 881 883 887 907 911 919 929 937 941
947 953 967 971 977 983 991 997 1009 1013
1019 1021 1031 1033 1039 1049 1051 1061 1063 1069
1087 1091 1093 1097 1103 1109 1117 1123 1129 1151
1153 1163 1171 1181 1187 1193 1201 1213 1217 1223
1229 1231 1237 1249 1259 1277 1279 1283 1289 1291
1297 1301 1303 1307 1319 1321 1327 1361 1367 1373
1381 1399 1409 1423 1427 1429 1433 1439 1447 1451
1453 1459 1471 1481 1483 1487 1489 1493 1499 1511
1523 1531 1543 1549 1553 1559 1567 1571 1579 1583
1597 1601 1607 1609 1613 1619 1621 1627 1637 1657
1663 1667 1669 1693 1697 1699 1709 1721 1723 1733
1741 1747 1753 1759 1777 1783 1787 1789 1801 1811
1823 1831 1847 1861 1867 1871 1873 1877 1879 1889
1901 1907 1913 1931 1933 1949 1951 1973 1979 1987
1993 1997 1999 2003 2011 2017 2027 2029 2039 2053
2063 2069 2081

%2083 2087 2089 2099 2111 2113 2129
%2131 2137 2141 2143 2153 2161 2179 2203 2207 2213
%2221 2237 2239 2243 2251 2267 2269 2273 2281 2287
%2293 2297 2309 2311 2333 2339 2341 2347 2351 2357
%2371 2377 2381 2383 2389 2393 2399 2411 2417 2423
%2437 2441 2447 2459 2467 2473 2477 2503 2521 2531
%2539 2543 2549 2551 2557 2579 2591 2593 2609 2617
%2621 2633 2647 2657 2659 2663 2671 2677 2683 2687
%2689 2693 2699 2707 2711 2713 2719 2729 2731 2741
%2749 2753 2767 2777 2789 2791 2797 2801 2803 2819
%2833 2837 2843 2851 2857 2861 2879 2887 2897 2903
%2909 2917 2927 2939 2953 2957 2963 2969 2971 2999
%3001 3011 3019 3023 3037 3041 3049 3061 3067 3079
%3083 3089 3109 3119 3121 3137 3163 3167 3169 3181
%3187 3191 3203 3209 3217 3221 3229 3251 3253 3257
%3259 3271 3299 3301 3307 3313 3319 3323 3329 3331
%3343 3347 3359 3361 3371 3373 3389 3391 3407 3413
%3433 3449 3457 3461 3463 3467 3469 3491 3499 3511
%3517 3527 3529 3533 3539 3541 3547 3557 3559 3571\\
\paragraph{Primos cercanos a $10^n$}\ \\
9941 9949 9967 9973 10007 10009 10037 10039 10061 10067 10069 10079\\
99961 99971 99989 99991 100003 100019 100043 100049 100057 100069\\
999959 999961 999979 999983 1000003 1000033 1000037 1000039\\
9999943 9999971 9999973 9999991 10000019 10000079 10000103 10000121\\
99999941 99999959 99999971 99999989 100000007 100000037 100000039 100000049\\
999999893 999999929 999999937 1000000007 1000000009 1000000021 1000000033

\paragraph{Cantidad de primos menores que $10^n$}\ \\
$\pi(10^1)$ = 4 ;
$\pi(10^2)$ = 25 ;
$\pi(10^3)$ = 168 ;
$\pi(10^4)$ = 1229 ;
$\pi(10^5)$ = 9592 \\
$\pi(10^6)$ = 78.498 ;
$\pi(10^7)$ = 664.579 ;
$\pi(10^8)$ = 5.761.455 ;
$\pi(10^9)$ = 50.847.534 \\
$\pi(10^{10})$ = 455.052,511 ;
$\pi(10^{11})$ = 4.118.054.813 ;
$\pi(10^{12})$ = 37.607.912.018% ;
%
% Fuente: http://primes.utm.edu/howmany.shtml#table
%
%
\subsubsection{Divisores}
Cantidad de divisores ($\sigma_0$) para \emph{algunos} $n / \neg\exists n'<n, \sigma_0(n') \geqslant \sigma_0(n)$ \\
$\sigma_0(60)$ = 12 ; $\sigma_0(120)$ = 16 ; $\sigma_0(180)$ = 18 ; $\sigma_0(240)$ = 20 ; $\sigma_0(360)$ = 24 \\
$\sigma_0(720)$ = 30 ; $\sigma_0(840)$ = 32 ; $\sigma_0(1260)$ = 36 ; $\sigma_0(1680)$ = 40 ; $\sigma_0(10080)$ = 72 \\ $\sigma_0(15120)$ = 80 ; $\sigma_0(50400)$ = 108 ; $\sigma_0(83160)$ = 128 ; $\sigma_0(110880)$ = 144 \\
$\sigma_0(498960)$ = 200 ; $\sigma_0(554400)$ = 216 ; $\sigma_0(1081080)$ = 256 ; $\sigma_0(1441440)$ = 288  $\sigma_0(4324320)$ = 384 ; $\sigma_0(8648640)$ = 448

%
Suma de divisores ($\sigma_1$) para \emph{algunos} $n / \neg\exists n'<n, \sigma_1(n') \geqslant \sigma_1(n)$ \\
$\sigma_1(96)$ = 252 ; $\sigma_1(108)$ = 280 ; $\sigma_1(120)$ = 360 ; $\sigma_1(144)$ = 403 ; $\sigma_1(168)$ = 480 \\
$\sigma_1(960)$ = 3048 ; $\sigma_1(1008)$ = 3224 ; $\sigma_1(1080)$ = 3600 ; $\sigma_1(1200)$ = 3844 \\
$\sigma_1(4620)$ = 16128 ; $\sigma_1(4680)$ = 16380 ; $\sigma_1(5040)$ = 19344 ; $\sigma_1(5760)$ = 19890 \\
$\sigma_1(8820)$ = 31122 ; $\sigma_1(9240)$ = 34560 ; $\sigma_1(10080)$ = 39312 ; $\sigma_1(10920)$ = 40320 \\
$\sigma_1(32760)$ = 131040 ; $\sigma_1(35280)$ = 137826 ; $\sigma_1(36960)$ = 145152 ; $\sigma_1(37800)$ = 148800 \\
$\sigma_1(60480)$ = 243840 ; $\sigma_1(64680)$ = 246240 ; $\sigma_1(65520)$ = 270816 ; $\sigma_1(70560)$ = 280098 \\
$\sigma_1(95760)$ = 386880 ; $\sigma_1(98280)$ = 403200 ; $\sigma_1(100800)$ = 409448  \\
$\sigma_1(491400)$ = 2083200 ; $\sigma_1(498960)$ = 2160576 ; $\sigma_1(514080)$ = 2177280 \\
$\sigma_1(982800)$ = 4305280 ; $\sigma_1(997920)$ = 4390848 ; $\sigma_1(1048320)$ = 4464096 \\
$\sigma_1(4979520)$ = 22189440 ; $\sigma_1(4989600)$ = 22686048 ; $\sigma_1(5045040)$ = 23154768 \\
$\sigma_1(9896040)$ = 44323200 ; $\sigma_1(9959040)$ = 44553600 ; $\sigma_1(9979200)$ = 45732192
%
%
\subsubsection{Factoriales}
\begin{tabular}{l|l}
0! =	1             & 11! = 39.916.800  \\
1! =	1             & 12! =	479.001.600	($\in \mathtt{int}$)\\
2! =	2             & 13! =	6.227.020.800	\\
3! =	6             & 14! =	87.178.291.200	\\
4! =	24            & 15! =	1.307.674.368.000	\\
5! =	120   			  & 16! =	20.922.789.888.000	\\
6! =	720           & 17! =	355.687.428.096.000	\\
7! =	5.040	        & 18! =	6.402.373.705.728.000	\\
8! =	40.320	      & 19! =	121.645.100.408.832.000	\\
9! =	362.880       & 20! =	2.432.902.008.176.640.000	($\in \mathtt{tint}$) \\
10! =	3.628.800     & 21! =	51.090.942.171.709.400.000
\end{tabular}

max signed tint = 9.223.372.036.854.775.807 \\
max unsigned tint = 18.446.744.073.709.551.615
%
%
\subsection{Soluci\'on de Sistemas Lineales}
\begin{code}
typedef vector<tipo> Vec;
typedef vector<Vec> Mat;
#define eps 1e-10
#define feq(a, b) (fabs(a-b)<eps)
bool resolver_ev(Mat a, Vec y, Vec &x, Mat &ev){
  int n = a.size(), m = n?a[0].size():0, rw = min(n, m);
  vector<int> p; forn(i,m) p.push_back(i);
  forn(i, rw){
    int uc=i, uf=i;
    // aca pivotea. lo unico importante es que a[i][i] sea no nulo
    forsn(f, i, n) forsn(c, i, m) if(fabs(a[f][c])>fabs(a[uf][uc])) {uf=f;uc=c;}
    if (feq(a[uf][uc], 0)) { rw = i; break; }
    forn(j, n) swap(a[j][i], a[j][uc]);
    swap(a[i], a[uf]); swap(y[i], y[uf]); swap(p[i], p[uc]);
    // fin pivoteo
    tipo inv = 1 / a[i][i]; //aca divide
    forsn(j, i+1, n) {
      tipo v = a[j][i] * inv;
      forsn(k, i, m) a[j][k]-=v * a[i][k];
      y[j] -= v*y[i];
    }
  } // rw = rango(a), aca la matriz esta triangulada
  forsn(i, rw, n) if (!feq(y[i],0)) return false; // checkeo de compatibilidad
  x = vector<tipo>(m, 0);
  dforn(i, rw){
    tipo s = y[i];
    forsn(j, i+1, rw) s -= a[i][j]*x[p[j]];
    x[p[i]] = s / a[i][i]; //aca divide
  }
  ev = Mat(m-rw, Vec(m, 0)); // Esta parte va SOLO si se necesita el ev
  forn(k, m-rw) {
    ev[k][p[k+rw]] = 1;
    dforn(i, rw){
      tipo s = -a[i][k+rw];
      forsn(j, i+1, rw) s -= a[i][j]*ev[k][p[j]];
      ev[k][p[i]] = s / a[i][i]; //aca divide
    }
  }
  return true;
}

bool diagonalizar(Mat &a){
	// PRE: a.cols > a.filas
	// PRE: las primeras (a.filas) columnas de a son l.i.
  int n = a.size(), m = a[0].size();
  forn(i, n){
		int uf = i;
		forsn(k, i, n) if (fabs(a[k][i]) > fabs(a[uf][i])) uf = k;
		if (feq(a[uf][i], 0)) return false;
		swap(a[i], a[uf]);
    tipo inv = 1 / a[i][i]; // aca divide
    forn(j, n) if (j != i) {
      tipo v = a[j][i] * inv;
      forsn(k, i, m) a[j][k] -= v * a[i][k];
		}
    forsn(k, i, m) a[i][k] *= inv;
  }
	return true;
}
\end{code}
\subsection{Programaci'on Lineal - Simplex}
\textbf{Teorema de dualidad (fuerte):} Dado un problema lineal $\Pi_1$: $\mathrm{minimizar}\ c^t\cdot X, \mathrm{sujeto\ a}\ A\cdot X \leqslant b, X \geqslant 0$ se define el problema lineal \emph{dual standard} $\Pi_2$  como: $\mathrm{minimizar}\ -b^t\cdot Y, \mathrm{sujeto\ a}\ A^t\cdot Y \geqslant c$. Si $\Pi_1$ es satisfacible entonces $\Pi_2$ es satisfacible y $c^t\cdot X = b^t \cdot Y$. Si $\Pi_1$ es insatisfacible o no acotado entonces $\Pi_2$ es insatisfacible o no acotado (Obs: no pueden ser ambos no acotados).

Dados \texttt{cfun}, \texttt{rmat} y \texttt{bvec}; Minimiza $\mathtt{cfun}^t \cdot \mathtt{xvar}$ sujeto a las condiciones $\mathtt{rmat} \cdot \mathtt{xvar} \leq \mathtt{bvec}$. Los valores de \texttt{bvec} pueden ser negativos para representar desigualdades de $\geq$ (por ejemplo: $-x \leq -5$).

Es sensible a errores num'ericos; se recomiendan valores de \texttt{eps=1e-16} y \texttt{epsval=1e-14}. El orden de magnitud de \texttt{epsval} debe ser del orden de la relaci'on entre los valores m'as grandes de \texttt{rmat}.
\begin{code}
usa: resolver,

#define MAXVAR 64
#define MAXRES 128
tipo rmat[MAXRES][MAXVAR+MAXRES*2];
tipo bvec[MAXRES];
tipo cfun[MAXVAR+MAXRES*2];
tipo xvar[MAXVAR];

#define HAYSOL	0
#define NOSOL -1
#define NOCOTA -2

int simplex(int m, int n) { // cant restric; cant vars
	int base[MAXVAR+MAXRES], esab[MAXVAR+MAXRES];
	int nn = n+m; // Variables (originales) + holgura
	tipo res = 0;

	forn(i, m) forn(j, m) rmat[i][n+j] = (i==j);
	forn(i, m) cfun[n+i] = 0;

	forn(i, n) esab[i] = -1;
	forn(i, m) { base[i] = n+i; esab[n+i] = i; }

	// Agrega las artificiales; si todos los bvec[] son positivos se puede omitir esto
	int arts[MAXRES];
	int bmin = 0;
	forn(i, m) if (bvec[i] < bvec[bmin]) bmin = i;
	int art = bvec[bmin] < -eps;
	forn(i, m) arts[i] = 2*(bvec[i] >= -eps) - 1;
	if (art) {
		forn(i, m) rmat[i][nn] = -(bvec[i] < -eps);
		esab[n+bmin] = -1; esab[nn] = bmin; base[bmin] = nn;
		nn++;
	}

	Mat B(m, Vec(m, 0));
	Vec y(m, 0), c(m, 0), d(m, 0);
	int j0 = 0;
	do {
		forn(i, m) forn(j, m) B[i][j] = arts[j] * rmat[j][base[i]];
		forn(i, m) c[i] = art?base[i]>=m+n:cfun[base[i]];
		resolver(B, c, y);

		for(; j0 < nn; ++j0) if (esab[j0] == -1) {
			res = art?j0>=m+n:cfun[j0];
			forn(i, m) res -= y[i] * arts[i] * rmat[i][j0];
			if (j0 < m+n && res < epsval) break;
		}

		forn(i, m) forn(j, m) B[i][j] = rmat[i][base[j]];
		forn(i, m) c[i] = rmat[i][j0];
		resolver(B, c, d);
		forn(i, m) c[i] = bvec[i];
		resolver(B, c, y);

		if (j0 == nn) if (art) {
			if (esab[m+n] != -1 && y[esab[m+n]] > epsval) return NOSOL;
			for(int i = m+n-1; i >= 0; i--) if (esab[i] == -1) { esab[i] = esab[m+n]; base[esab[i]] = i; break; }
			art = 0; nn = m+n; j0 = 0; continue;
		} else break; // Optimo

		bool bl = true;
		forn(i, m) bl = bl && (d[i] <= eps);

		if (bl) return NOCOTA; // Problema no acotado

		int j1 = 0;
		forn(i, m) if (d[i] > 0) {
			tipo mlt = y[i] / d[i];
			if (!bl || (feq(mlt, res) && (base[i] < j1)) || (mlt < res)) {
				res = mlt;
				j1 = base[i];
				bl = true;
			}
		}
		if (res < eps && ++j0) continue;
		if (art && j1 == m+n) nn--, art--;

		int w = esab[j1];      // variable de salida
		base[w] = j0;           // Entra j0
		esab[j0] = w;
		esab[j1] = -1;         // j1 es no basica ahora.
		j0 = 0;
	} while(1);

	forn(i, m) forn(j, m) B[i][j] = rmat[i][base[j]];
	forn(i, m) c[i] = bvec[i];
	resolver(B, c, y);

	forn(i, n) xvar[i] = (esab[i] == -1)?0:y[esab[i]];

	return HAYSOL;
}
\end{code}
\subsection{Factorizaci'on QR de Householder}
Descompone $A = Q\cdot R$. Observaci'on: $|det(A)| = |det(R)|$.
\begin{code}
typedef vector<vector<tipo> > Mat;
typedef vector<tipo> Vec;
tipo sqr(tipo x) {return x*x;}

void show(Mat &a);

void qr(const Mat &a, Mat &q, Mat &r) {
	int n = a.size();
	r = a;
	q = Mat(n, Vec(n, 0));
	forn(i, n) forn(j, n) q[i][j] = (i==j);

	forn(k, n-1) {
		tipo beta = 0;
		forsn(i, k, n) beta += sqr(r[i][k]);
		tipo alph = sqrt(beta);
		if (alph * r[k][k] >= 0) alph = -alph;

		Vec v(n, 0);
		forsn(i, k, n) v[i] = r[i][k]; v[k] -= alph;
		beta += sqr(v[k]) - sqr(r[k][k]);

		#define QRmult(X) \
		forn(i, n) { tipo w = 0; \
			forsn(j, k, n) w += X * v[j]; w /= beta/2; \
			forsn(j, k, n) X -= w * v[j]; }

		// Q := Q * (I - 2 v * v^t) = Q - 2 * ((Q * v) * v^t)
		QRmult(q[i][j]);
		// A := Qj * A; \equiv A^t := A^t * Qj;
		QRmult(r[j][i]);

		forsn(i, k+1, n) r[i][k] = 0;
	}
}

// QR para calcular autvalores (no estoy seguro de para qu� matrices sirve)
Mat operator* (const Mat &ml, const Mat &mr) {
	int a = ml.size(), b = mr.size(), c = mr[0].size();
	Mat res(a, Vec(c, 0));
	forn(i, a) forn(j, c) forn(k, b) res[i][j] += ml[i][k] * mr[k][j];
	return res;
}

#define iterac ???
void autoval(Mat &a) {
	int n = a.size();
	Mat q(n, Vec(n, 0));
	forn(i, iterac) {
		qr(a, q, a);
		a = a * q;
	}
	// Los autovalores convergen en la diagonal de "a"
}
\end{code}
\subsection{Multiplicaci'on de Karatsuba}
\texttt{BASE} y \texttt{BASEXP} deben ser tales que $\mathtt{BASE} = 10^\mathtt{BASEXP}$ y adem'as, $\mathtt{BASE}^2\cdot largo$ entre en un \texttt{int} o \texttt{tint}, seg'un el caso.\\
Los n'umeros se representan en base \texttt{BASE} con la parte menos significativa en los 'indices m'as bajos.
\begin{code}
#define BASE 1000000
#define BASEXP 6

typedef tint tipo; // o int

tipo* ini(int l){
	tipo *r = new tipo[l];
	fill(r, r+l, 0);
	return r;
}
#define add(l,s,d,k)forn(i, l)(d)[i]+=(s)[i]*k
void mulFast(int l, tipo *n1, tipo *n2, tipo *nr){
	if(l<=0)return;
	if(l<35){
		forn(i, l)forn(j, l)nr[i+j]+=n1[i]*n2[j];
	}else{
		int lac = l/2, lbd = l - (l/2);
		tipo *a = n1, *b=n1+lac, *c=n2, *d=n2+lac;
		tipo *ab = ini(lbd+1), *cd = ini(lbd+1);
		tipo *ac = ini(lac+lac), *bd = ini(lbd+lbd);
		add(lac, a, ab, 1);
		add(lbd, b, ab, 1);
		add(lac, c, cd, 1);
		add(lbd, d, cd, 1);
		mulFast(lac, a, c, ac);
		mulFast(lbd, b, d, bd);
		add(lac+lac, ac, nr+lac,-1);
		add(lbd+lbd, bd, nr+lac,-1);
		add(lac+lac, ac, nr,1);
		add(lbd+lbd, bd, nr+lac+lac,1);
		mulFast(lbd+1, ab, cd, nr+lac);
		free(ab); free(cd); free(ac); free(bd);
	}
}
void mulFast(int l1, tipo *n1, int l2, tipo *n2, int &lr, tipo *nr){
	while(l1<l2) n1[l1++]=0;
	while(l2<l1) n2[l2++]=0;
	lr=l1+l2+3;
	fill(nr, nr+lr, 0);
	mulFast(l1, n1, n2, nr);

	tipo r = 0;
	forn(i, lr){
		tipo q = r+nr[i];
		nr[i] = q%BASE,r = q/BASE;
	}
	while(lr>1 && nr[lr-1]==0)lr--;
}

// Cosas extra (convierten entre base 10 y 10^n)
void base10ton(int &l, tipo* n) {
	tipo p10[BASEXP]; p10[0] = 1;
	forn(i, BASEXP-1) p10[i+1] = p10[i] * 10;

	int nl = (l+BASEXP-1)/BASEXP;
	forsn(i, l, nl*BASEXP) n[i] = 0;
	forn(i, nl) {
		tint s = 0;
		forn(j, BASEXP) s+= n[i*BASEXP+j]*p10[j];
		n[i] = s;
	}
	l = nl;
}

void baseNto10(int &l, tipo* n) {
	for(int i = l-1; i>=0; --i) {
		tipo v = n[i];
		forn(j, BASEXP) {
			n[i*BASEXP+j] = v % 10; v /= 10;
		}
	}
	l = l*BASEXP;
	while (!n[l-1] && l > 1) l--;
}
\end{code}
%
\subsection{Long - Entero largo}\begin{code}
typedef tint tipo;
#define BASEXP 6
#define BASE 1000000
#define LMAX 1000

struct Long {
	int l;
	tipo n[LMAX];
	Long(tipo x) { 	l = 0; forn(i, LMAX) { n[i]=x%BASE; l+=!!x||!i; x/=BASE;} }
	Long(){*this = Long(0);}
	Long(string x) {
		l=(x.size()-1)/BASEXP+1;
		fill(n, n+LMAX, 0);
		tipo r=1;
		forn(i,x.size()){
			n[i / BASEXP] += r * (x[x.size()-1-i]-'0');
			r*=10; if(r==BASE)r=1;
		}
	}
};

void out(Long& a) {
	char msg[BASEXP+1];
	cout << a.n[a.l-1];
	dforn(i,a.l-1) {
		sprintf(msg, "%6.6llu", a.n[i]); cout << msg; // 6 = BASEXP !
	}
	cout << endl;
}
void invar(Long &a) {
	fill(a.n+a.l, a.n+LMAX, 0);
	while(a.l>1 && !a.n[a.l-1]) a.l--;
}

void lsuma(const Long&a, const Long&b, Long&c) { // c = a + b
	c.l = max(a.l, b.l);
	tipo q = 0;
	forn(i, c.l) q += a.n[i]+b.n[i], c.n[i]=q%BASE, q/=BASE;
	if(q) c.n[c.l++] = q;
	invar(c);
}
Long& operator+= (Long&a, const Long&b) { lsuma(a, b, a); return a; }
Long operator+ (const Long&a, const Long&b) { Long c; lsuma(a, b, c); return c; }

bool lresta(const Long&a, const Long&b, Long&c) { // c = a - b
	c.l = max(a.l, b.l);
	tipo q = 0;
	forn(i, c.l) q += a.n[i]-b.n[i], c.n[i]=(q+BASE)%BASE, q=(q+BASE)/BASE-1;
	invar(c);
	return !q;
}
Long& operator-= (Long&a, const Long&b) { lresta(a, b, a); return a; }
Long operator- (const Long&a, const Long&b) {Long c; lresta(a, b, c); return c;}

bool operator< (const Long&a, const Long&b) { Long c; return !lresta(a, b, c); }
bool operator<= (const Long&a, const Long&b) { Long c; return lresta(b, a, c); }
bool operator== (const Long&a, const Long&b) { return a <= b && b <= a; }

void lmul(const Long&a, const Long&b, Long&c) { // c = a * b
	c.l = a.l+b.l;
	fill(c.n, c.n+b.l, 0);
	forn(i, a.l) {
		tipo q = 0;
		forn(j, b.l) q += a.n[i]*b.n[j]+c.n[i+j], c.n[i+j] = q%BASE, q/=BASE;
		c.n[i+b.l] = q;
	}
	invar(c);
}

Long& operator*= (Long&a, const Long&b) { Long c; lmul(a, b, c); return a=c; }
Long operator* (const Long&a, const Long&b) { Long c; lmul(a, b, c); return c; }

void lmul(const Long&a, tipo b, Long&c) { // c = a * b
	tipo q = 0;
	forn(i, a.l) q += a.n[i]*b, c.n[i] = q%BASE, q/=BASE;
	c.l = a.l;
	while(q) c.n[c.l++] = q%BASE, q/=BASE;
    invar(c);
}

Long& operator*= (Long&a, tipo b) { lmul(a, b, a); return a; }
Long operator* (const Long&a, tipo b) { Long c = a; c*=b; return c; }

void ldiv(const Long& a, tipo b, Long& c, tipo& rm) { // c = a / b ; rm = a % b
	rm = 0;
	dforn(i, a.l) {
		rm = rm * BASE + a.n[i];
		c.n[i] = rm / b; rm %= b;
	}
	c.l = a.l;
	invar(c);
}

Long operator/ (const Long&a, tipo b) { Long c; tipo r; ldiv(a, b, c, r); return c; }
tipo operator% (const Long&a, tipo b) { Long c; tipo r; ldiv(a, b, c, r); return r; }

void ldiv(const Long& a, const Long& b, Long& c, Long& rm) { // c = a / b ; rm = a % b
	rm = 0;
	dforn(i, a.l) {
        if (rm.l == 1 && rm.n[0] == 0)
            rm.n[0] = a.n[i];
        else {
            dforn(j, rm.l) rm.n[j+1] = rm.n[j];
            rm.n[0] = a.n[i]; rm.l++;
        }
		tipo q = rm.n[b.l] * BASE + rm.n[b.l-1];
		tipo u = q / (b.n[b.l-1] + 1);
		tipo v = q /  b.n[b.l-1] + 1;
		while (u < v-1) {
			tipo m = (u+v)/2;
			if (b*m <= rm) u = m; else v = m;
		}
		c.n[i] = u;
		rm -= b*u;
	}
	c.l = a.l;
	invar(c);
}

Long operator/ (const Long&a, const Long & b) { Long c,r; ldiv(a, b, c, r); return c; }
Long operator% (const Long&a, const Long & b) { Long c,r; ldiv(a, b, c, r); return r; }
\end{code}
\subsection{Fracci'on}
\begin{code}
usa: algorithm, tint, mcd
struct frac {
	tint p,q;
	frac(tint num=0, tint den=1):p(num),q(den) { norm(); }
	frac& operator+=(const frac& o){
		tint a = mcd(q,o.q);
		p=p*(o.q/a)+o.p*(q/a);
		q*=(o.q/a);
		norm();
		return *this;
	}
	frac& operator-=(const frac& o){
		tint a = mcd(q,o.q);
		p=p*(o.q/a)-o.p*(q/a);
		q*=(o.q/a);
		norm();
		return *this;
	}
	frac& operator*=(frac o){
		tint a = mcd(q,o.p);
		tint b = mcd(o.q,p);
		p=(p/b)*(o.p/a);
		q=(q/a)*(o.q/b);
		return *this;
	}
	frac& operator/=(frac o){
		tint a = mcd(q,o.q);
		tint b = mcd(o.p,p);
		p=(p/b)*(o.q/a);
		q=(q/a)*(o.p/b);
		norm();
		return *this;
	}

	void norm(){
		tint aux = mcd(p,q);
		if (aux){ p/=aux; q/=aux; }
		else { q=1; }
		if (q<0) { q=-q; p=-p; }
	}
};
\end{code}
\newpage
\section{Cosas}\subsection{Morris-Prath}
premp$[i+1]$ da el maximo $k$ en $[0,i)$ tal que $s[0,k) = s[i-k,i)$
\begin{code}
tint pmp[MAXL];
void preMp(string& x){
  tint i=0, j = pmp[0] = -1;
  while(i<(tint)x.size()){
    while(j>-1 && x[i] != x[j]) j = pmp[j];
      pmp[++i] = ++j;
  }
}
void mp(string& b, string& g){
  preMp(b);
  tint i=0,j=0;
  while(j<(tint)g.size()){
    while(i>-1 && b[i] != g[j]){i = pmp[i];}
    i++; j++;
    if (i>=(tint)b.size()){
      OUTPUT(j - i);
      i=pmp[i];
    }
  }
}
\end{code}
\subsection{Subsecuencia com'un m'as larga}\begin{code}
tint lcs(vector<tint> a, vector<tint> b) { // Longest Common Subsequence
	vector< vector<tint> > m(2, vector<tint>(b.size()+1));
	forn(i,a.size())forn(j,b.size())
    m[1-i%2][j+1]=(a[i]==b[j]?m[i%2][j]+1:max(m[i%2][j+1],m[1-i%2][j]));
	return m[a.size()%2][b.size()];
}
\end{code}
\subsection{SAT - 2}
\begin{code}
usa: stack
#define MAXN 1024
#define MAXEQ 1024000

int fch[2*MAXN], nch[2*MAXEQ], dst[2*MAXEQ], eqs; // Grafo
#define addeje(s,d) { nch[eqs]=fch[s]; dst[fch[s]=eqs++]=d; }
#define neg(X) (2*MAXN-1-(X))
void init() {
	memset(fch, 0xff, sizeof(fch));
	eqs=0;
}
void addEqu(int a, int b) {
	addeje(neg(a), b);
	addeje(neg(b), a);
}
char verdad[2*MAXN]; // Solo si interesa el valor de verdad
int us[2*MAXN], lw[2*MAXN], id[2*MAXN];
stack<int> q; int qv, cp;
void tjn(int i) {
	lw[i] = us[i] = ++qv;
	id[i]=-2; q.push(i);
	for(int j = fch[i]; j!=-1; j=nch[j]) { int x = dst[j];
		if (!us[x] || id[x] == -2) {
			if (!us[x]) tjn(x);
			lw[i] = min(lw[i], lw[x]);
		}
	}
	if (lw[i] == us[i]) {
		int x; do { x = q.top(); q.pop(); id[x]=cp; } while (x!=i);
        verdad[cp] = (id[neg(i)] < 0); // Valor de verdad de variable i es verdad[id[i]]
		cp++;
	}
}
void compCon(int n) { // Tarjan algorithm
	memset(us, 0, sizeof(us));
	memset(id, -1, sizeof(id));
	q=stack<int>(); qv = cp = 0;
	forn(i, n) {
		if (!us[i]) tjn(i);
		if (!us[neg(i)]) tjn(neg(i));
	}
}
bool satisf(int n) {
	compCon(n);
	forn(i, n) if (id[i] == id[neg(i)]) return false;
	return true;
}
\end{code}
\subsection{Male-optimal stable marriage problem O($N^2$)}
\verb+gv[i][j]+ es la \textit{j}-esima mujer en orden de preferencia en la lista del varon \textit{i}. \\
\verb+om[i][j]+ es la posici'on que ocupa el hombre \textit{j} en la lista de la mujer \textit{i}.
\begin{code}
#define MAXN 1000
int gv[MAXN][MAXN],om[MAXN][MAXN]; // Inpu del algoritmo
int pv[MAXN],pm[MAXN];             // Oupu del algoritmo
int pun[MAXN];                     // Auxiliar

void stableMarriage(int n) {
	fill_n(pv,n,-1); fill_n(pm,n,-1); fill_n(pun,n,0);
	int s = n, i = n-1;
	#define engage pm[j] = i; pv[i] = j;
	while (s) {
		while (pv[i] == -1) {
			int j = gv[i][pun[i]++];
			if (pm[j] == -1) {
				s--; engage;
			}
			else if (om[j][i] < om[j][pm[j]]) {
				int loser = pm[j];
				pv[loser] = -1;
				engage;
				i = loser;
			}
		}
		i--; if (i < 0) i = n-1;
} }
\end{code}
\subsection{Rotaciones del cubo}
\begin{code}
#define _ALTA {forn(h, 6) rot[p][h] = d[h]; p++;}
#define _DER forn(h, 6) d[h] = _der[d[h]];
#define _UP forn(h, 6) d[h] = _up[d[h]];

int rot[24][6];
const int _der[6] = {0, 2, 4, 1, 3, 5};
const int _up[6] = {1, 5, 2, 3, 0, 4};

void rotaciones() {
	int d[6];
	int p = 0;
	forn(i, 6) d[i] = i;                  / \
	forn(i, 2) {                   4 --> /   \ <-- 3
		forn(j, 3) {                      |\ 0 /|
			_ALTA; _DER;                    | \ / |
			_ALTA; _DER;                    |2 | 1|
			_ALTA; _DER;                     \ | /
			_ALTA; _UP;                       \|/
		}                                    ^
		_DER; _UP; _UP;                      5
	}
	return;
}
\end{code}
\subsection{Poker}
\begin{code}
usa: list, vector, map, string, algorithm, forn, tint, pint
#define STRAIGHT_VALUE 14
#define FLUSH_VALUE 15
typedef pair<int,int> pint;
typedef vector< pint > hand;
typedef vector< int > puntaje;

int cantPairs(hand& m) {
    int pares=0;
    forn(i,m.size()) forn(j,i) if (m[i].first == m[j].first)
    pares++;
    return pares;
}

int isStraight(hand& m) {
    sort(m.begin(), m.end());
    int ls=4;
    if (m[4].first==14 && m[0].first==2) ls=3; //esta linea acepta escaleras desde el A
    forn(i, ls) if (m[i].first != m[i+1].first - 1) return 0;
    return STRAIGHT_VALUE;
}

int isFlush(hand& m) {
    forn(i, m.size()-1) if (m[i].second != m[i+1].second) return 0;
    return FLUSH_VALUE;
}

int gamePoints(hand& m) {
    int f=isFlush(m),s=isStraight(m),p=cantPairs(m) * 4;
    return max(f+s,p); //esto esta para aceptar cartas duplicadas
}

puntaje points(hand& m) {
    puntaje r;
    r.push_back(gamePoints(m));
    map<int, int> c;
    int i;
    forn(i,m.size()) c[m[i].first]++;
    vector<pint> cants;
    map<int, int>::iterator it;
    for(it = c.begin() ; it != c.end() ; ++it) {
        cants.push_back( pint( it->second, it->first ) );
    }
    sort(cants.begin(), cants.end());
    forn(i, cants.size()) {
        r.push_back(cants[cants.size()-1-i].second);
    }
    //esta linea que sigue arregla la comparacion con escaleras que empiezan desde A
    if ((r[0]==FLUSH_VALUE || r[0]==FLUSH_VALUE+STRAIGHT_VALUE) && r[1]==14 && r[2]!=13) r[1]=1;
    return r;
}
tint comp(hand& m1, hand& m2) {
  puntaje n1 = points(m1); puntaje n2 = points(m2);
  return (n1 > n2 ? 1 : n1 == n2 ? 0 : -1);
}
tint convN(char c) {
  switch(c) {
  case 'A': return 14; case 'K': return 13; case 'Q': return 12;
  case 'J': return 11; case 'T': return 10;
  } return c - '0';
}
pint readCard() {
  string s; cin >> s;
  return (s == "" ? pint(-1,-1) : pint(convN(s[0]), s[1]));
}
hand readHand() { hand r;
  forn(i,5) {
    pint c = readCard();
    if (c == pint(-1,-1)) return hand();
    r.push_back(c);
  } return r;
}
\end{code}
\section{Extras}
\subsection{Convex Hull en 3D}
Le das un mar de puntos y un triangulito inicial en una cara de la convex hull.
\begin{code}
usa: cstdio, vector, queue, iostream, fstream, cmath

const double KETO = 1e-9;
typedef long double tdbl;
inline tint sqr(tint a){return a*a; }
struct pto{tint x,y,z;};
pto point(tint x, tint y, tint z){pto r; r.x=x; r.y=y;r.z=z; return r;}

pto operator - (pto a, pto b) { return point(a.x-b.x, a.y-b.y, a.z-b.z); }
pto operator ^ (pto a, pto b) { return point(a.y*b.z-a.z*b.y,
a.z*b.x-a.x*b.z, a.x*b.y-a.y*b.x); }
tint operator * (pto a, pto b) { return a.x*b.x + a.y*b.y + a.z*b.z; }
bool operator == (pto a, pto b) { return a.x==b.x && a.y==b.y && a.z==b.z; }
tdbl len (pto a) { return sqrt(1.0*(a*a)); }
tint len2(pto a) { return a*a; }
ifstream in("d.in");
ifstream out("d.out");
#define FS first
#define SD second
#define MP make_pair
bool ok[1700][1700];
int main () {
	int runs; in >> runs;
	while (runs--) {
		vector<pto> p;
	    int x1,y1,x2,y2;
	    in >> x1 >> y1 >> x2 >> y2;
	    p.push_back(point(x1,y1,0)); p.push_back(point(x2,y1,0));
      p.push_back(point(x2,y2,0)); p.push_back(point(x1,y2,0));
		int N; in >> N;
		tdbl area=-abs(x2-x1)*abs(y2-y1), area2;
		if(N){
			forn(i, N){
			    int h; in >> x1 >> y1 >> x2 >> y2 >> h;
			    p.push_back(point(x1,y1,h)); p.push_back(point(x2,y1,h));
          p.push_back(point(x2,y2,h)); p.push_back(point(x1,y2,h));
			}
			fill(ok[0], ok[p.size()], false);
			queue<pair<pair<int, int>, int> > q;
			q.push(MP(MP(0,1),2));
			while (!q.empty()) {
				int A = q.front().FS.FS;
				int B = q.front().FS.SD;
				int x = q.front().SD; q.pop();
				if (ok[A][B]) continue;
				tdbl Ccos3D = -1e100;
				tdbl Ccos2D = -1e100;
				tdbl Cdist  = -1e100;
	  			int C = -1;
				pto n = (p[x]-p[B]) ^ (p[x]-p[A]);
				forn(i, p.size()){
					if (ok[B][i] || ok[i][A]) continue;
					pto mi = (p[i]-p[A]) ^ (p[i]-p[B]);
					if (mi.x==0&&mi.y==0&&mi.z==0) continue;

					tdbl icos3D = tdbl (mi*n) / len(mi) / len(n);
					tdbl icos2D = tdbl ((p[i]-p[B])*(p[B]-p[A])) / len(p[i]-p[B]) / len(p[B]-p[A]);
					tdbl idist  = len(mi);

					if ((icos3D>Ccos3D+KETO) ||
					    (icos3D>Ccos3D-KETO && icos2D>Ccos2D+KETO) ||
					    (icos3D>Ccos3D-KETO && icos2D>Ccos2D-KETO && Cdist<idist)) {
						C = i;
						Ccos3D = icos3D;
						Ccos2D = icos2D;
						Cdist  = idist;
					}
				}
				ok[A][B]=ok[B][C]=ok[C][A]=true;
				q.push(MP(MP(C,B), A));
				q.push(MP(MP(A,C), B));
				area += 0.5 * len((p[C]-p[A]) ^ (p[C]-p[B]));
			}
		}else{
			area = -area;
		}
		out >> area2;
		if(abs(area2-area)>1e-4){
			cout << "MAL" << endl;
		}
	}
	cout << "FIN" << endl;
	return 0;
}
\end{code}
\subsection{Componentes conexas en un subgrafo grilla}
\begin{code}
int dx[4]={0,0,-1,1}, dy[4]={-1,1,0,0};
struct Cas{int p[4];};
const int MAXN = 105;
Cas c[MAXN*2][MAXN*2];
int px, py;
void put(int x, int y, int d, int l, int t){
	forn(i, l){
		if(d==0)c[x+i][y].p[0] = c[x+i][y-1].p[1] = t;
		if(d==1)c[x][y+i].p[2] = c[x-1][y+i].p[3] = t;
	}
}
void init(){
	Cas cc; fill(cc.p, cc.p+4, 0);
	fill(c[0], c[MAXN], cc);
}
\end{code}
\subsection{Orden total de puntos alrededor de un centro}
\begin{code}
struct Cmp{
	pto r;
	Cmp(pto _r){r = _r;}
	int cuad(const pto &a) const{
		if(a.x > 0 && a.y >= 0)return 0;
		if(a.x <= 0 && a.y > 0)return 1;
		if(a.x < 0 && a.y <= 0)return 2;
		if(a.x >= 0 && a.y < 0)return 3;
		assert(a.x ==0 && a.y==0);
		return -1;
	}
	bool cmp(const pto&p1, const pto&p2)const{
		int c1 = cuad(p1), c2 = cuad(p2);
		if(c1==c2){
			return p1.y*p2.x<p1.x*p2.y;
		}else{
			return c1 < c2;
		}
	}
	bool operator()(const pto&p1, const pto&p2) const{
		return cmp(pto(p1.x-r.x,p1.y-r.y),pto(p2.x-r.x,p2.y-r.y));
  }
};
\end{code}
\section{El AJI es una fruta}
\subsection{Dinitz}
\begin{code}
#define INF 1000000000 // Infinito de FLUJO
#define DINF INF       // Infinito de distancia
#define MAX_M 1000000  // Maximo de aristas
#define MAX_N 45000    // Maximo de nodos
int v[2*MAX_M], l[2*MAX_M];	// Vecino, link. link te tira el indice de la arista "al reves" asociada en la lista del vecino.
long c[2*MAX_M];	// Capacidad
int sz[MAX_N], po[MAX_N], r[MAX_N], n, S, T;
typedef map<int,long> Mii;
Mii CAP[MAX_N];
void iniG() {
	n = 0;
	memset(sz,0,sizeof(sz));
	forn(i,MAX_N) CAP[i].clear();
}
void aEje(int d,int h,long cap) {
	if (d == h) return; // Ignoramos completamente autoejes, obvio :D
	n = max(n,max(d,h));
	pair<Mii::iterator,bool> par = CAP[d].insert(make_pair(h,0));
	if (par.second) {
		CAP[h][d] = 0;
		sz[d]++;
		sz[h]++;
	}
	par.first->second += max(cap,(long)0);
}
void _aEje(int d,int h,long capDH, long capHD) {
	#define ASIG(d,h,cap) {v[po[d]] = h; c[po[d]] = cap; l[po[d]] = po[h];}
	ASIG(d,h,capDH);
	ASIG(h,d,capHD);
	po[d]++; po[h]++;
}
void _iniG() {
	po[0] = 0;
	forn(i,n-1) po[i+1] = po[i] + sz[i];
	forn(u,n) forall(v,CAP[u])
		if (u < v->first) _aEje(u,v->first,v->second,CAP[v->first][u]);
}
long aumentar() {
	// bfs
	forn(i,n) r[i] = DINF;
	r[T] = 0;
    static int q[MAX_N];
    int qf = 0, qb = 0;
	q[qb++] = T;
	while (qb != qf) {
		int x = q[qf++];
		int d = r[x] + 1, b = po[x];
		if (r[S] < DINF) break;
		forsn(j,b,b+sz[x])
		if (c[l[j]]>0 && r[v[j]] == DINF) {
			r[v[j]] = d;
			q[qb++] = v[j];
		}
	}
	// dfs que hace la magia :P
	long res = 0;
	static int path[MAX_N]; path[0] = S;
	static int p[MAX_N],ind[MAX_N];
    forn(i,n) p[i] = -1;
	int pp = 0; // Path pointer, es la longitud
	while (pp >= 0)	{
		int x = path[pp];
		if (x == T)	{ // Llegamo, hay que hacer magia. O sea, ajustar todas las capacidades a lo largo del caminito que se satura.
			long f = INF;
			int pri = 0;
			dforn(i,pp)	if (c[ind[i]]<=f) f = c[ind[i]], pri = i;
			forn(i,pp) c[ind[i]] -= f, c[l[ind[i]]] += f;
			res += f;
			pp = pri;
		}
		else if (++p[x] < sz[x]) {
			int j = po[x]+p[x];
			if (p[v[j]] < 0 && c[j] > 0 && r[v[j]] < r[x])
				ind[pp] = j, path[++pp] = v[j];
		}
		else pp--;
	}
	return res;
}
long flujo(int ss,int tt) {
	S = ss; T = tt;
	n = max(n,max(S,T)) + 1; // Aca, n ya tiene el valor posta
	_iniG();
	forn(i,n) po[i] -= sz[i];
	
	long res = 0,c;
	do {res += (c = aumentar());} while (c>0);
	return res;
}
\end{code}
\subsection{FFT}
\begin{code}
typedef double tipo;
// Tipo debe ser un tipo de punto flotante

struct Comp
{
	tipo r,i;
	Comp() : r(0), i(0) {}
	Comp(tipo rr) : r(rr), i(0) {}
	Comp(tipo rr, tipo ii) : r(rr), i(ii) {}
	Comp operator + (const Comp &o) const { return Comp(r + o.r, i + o.i); }
	Comp operator - (const Comp &o) const {	return Comp(r - o.r, i - o.i); }
	Comp operator * (const Comp &o) const {	return Comp(r * o.r - i * o.i,r * o.i + o.r * i); }
	Comp & operator += (const Comp &o) { return *this = *this + o; }
	Comp & operator *= (const Comp &o) { return *this = *this * o; }
};

#define MAXN (1<<21)	// Debe ser potencia de 2
						// Si se esta usando convolucion, debe ser
						// al menos 2 * n

const tipo PI = 4.0 * atan(1.0);

unsigned bTabla[(1<<16)]; 	// Es importante que sea unsigned para que al hacer shift right
							// haga shift logico y no shift aritmetico.

// Se llama a llenarTabla una vez al comienzo del programa.

void llenarTabla()
{
	bTabla[0] = 0;
	forn(i,16)
	{
		int s = (1<<i);
		forn(j,s) bTabla[s + j] = (bTabla[j] *= 2)+1;
	}
}

struct Fft
{
	// Tipo debe ser un tipo de punto flotante
	int n,nk;
	Comp v[MAXN];
	Comp *init(int nn)
	{
		for(n = 1, nk = 32; n < nn; n *= 2, nk--);
		zMem(v);
		return v;
	}
	Comp *fft()
	{
		forn(i,n)
		{
			int j = (((bTabla[i&0xFFFF] << 16) | bTabla[i>>16]) >> nk);
			if (j > i) swap(v[i],v[j]);
		}
		for(int i=2;i<=n; i+=2)
		{
			int s = 1, x = i;
			while (x%2 == 0)
			{
				x /= 2;
				Comp w1 = 1.0;
				const Comp w(cos(PI/tipo(s)),-sin(PI/tipo(s)));
				Comp *A = v+i-2*s, *B = v+i-s, *FIN = v+i;
				// Cuello de botella : este for.
				// En particular, las dos multiplicaciones.
				for(;B != FIN;A++,B++)
				{
					Comp X = *A, Y = w1 * *B;
					*A += Y;
					*B = X - Y;
					w1 *= w;
				}
				s *= 2;
			}
		}
		
		return v;
	}
	Comp *invfft()
	{
		reverse(v+1,v+n);
		forn(i,n) v[i] *= 1.0 / tipo(n);
		return fft();
	}
	// Uso:
	// Se llama a init, se llena el vector en el puntero devuelto.
	// Se llama a fft o a invftt, segun lo buscado.
	// El output aparece en el puntero devuelto.
};

struct Convolucion
{
	// Tipo debe ser un tipo de punto flotante
	Fft fft;
	Comp v[MAXN]; // Se usa el MAXN de fft
	Comp* init(int n)
	{
		return fft.init(2*n);
	}
	Comp *next()
	{
		fft.fft();
		memcpy(v,fft.v,sizeof(v));
		zMem(fft.v);
		return fft.v;
	}
	Comp *conv()
	{
		fft.fft();
		forn(i,fft.n)
			fft.v[i] *= v[i];
		return fft.invfft();
	}
	// Uso:
	// Se llama a init, se llena el primer vector en el puntero devuelto.
	// Se llama a next, se llena el segundo vector en el puntero devuelto.
	// Se llama a conv, la convolucion aparece en el puntero devuelto.
};
\end{code}
\subsection{Intersecci'on (y yerbas afines) de circulos en $O(n^3 \lg n)$}
\begin{code}
typedef double real; // abstraccion magica

struct pto
{
	real x,y;
	pto() : x(0),y(0) {}
	pto(real xx, real yy) : x(xx),y(yy) {}
	pto operator +(const pto &o) const { return pto(x+o.x,y+o.y); }
	pto operator -(const pto &o) const { return pto(x-o.x,y-o.y); }
	pto operator *(real k) const { return pto(k*x,k*y); }
	real norma() const { return hypot(x,y); }
	pto rotar(real alfa) const { return pto(x * cos(alfa) - y * sin(alfa), x* sin(alfa) + y * cos(alfa)); }
};

struct circ { pto c; real r; };

#define sqr(x) ((x)*(x))

struct event
{
	real x; int t; 
	event(real xx, int tt) : x(xx), t(tt) {}
	bool operator <(const event &o) const { return x < o.x; }
};

typedef vector<circ> VC;
typedef vector<event> VE;

real cuenta(VE &v, real A,real B)
{
	sort(all(v));
	real res = 0.0, lx = ((v.empty())?0.0:v[0].x);
	int contador = 0;
	forn(i,v.size())
	{
		// Esta es la linea magica que hay que tocar.
		// Cambiando trivialmente el if, hacemos que compute interseccion de todos (contador == n),
		// union de todos (contador > 0), conjunto de puntos cubierto por exactamente k circulos (contador == k),
		// etc. En este caso, le estamos pidiendo los puntos que son tocados por 1,2 o 3
		// circulos, que es lo que queremos pal problema del robotito que tira rayitos :D
		if (contador > 0 && contador < 4) res += v[i].x - lx;
		contador += v[i].t;
		lx = v[i].x;
	}
	return res;
}

const real PI = 4.0 * atan(1.0);

// La siguiente da una primitiva de sqrt(r*r - x*x) como funcion real de una variable x.
// Los bordes estan puestos estrategicamente para que todo ande joya :D
inline real primitiva(real x,real r)
{
	if (x >= r) return r*r*PI/4.0;
	if (x <= -r) return -r*r*PI/4.0;
	real raiz = sqrt(r*r-x*x);
	return 0.5 * (x * raiz + r*r*atan(x/raiz));
}

// Se llama asi pero en realidad calcula la funcion que calcule "cuenta" en base a los "intervalos" que esta le arma.
// Puede ser interseccion, union, o incluso algunas cosas mas locas :D.
real interCirc(const VC &v)
{
	vector<real> p; p.reserve(v.size() * (v.size() + 2));
	forn(i,v.size())
	{
		p.push_back(v[i].c.x + v[i].r);
		p.push_back(v[i].c.x - v[i].r);
	}
	forn(i,v.size())
	forn(j,i)
	{
		const circ &a = v[i], b = v[j];
		real d = (a.c - b.c).norma();
		if (fabs(a.r - b.r) < d && d < a.r + b.r)
		{
			real alfa = acos((sqr(a.r) + sqr(d) - sqr(b.r)) / (2.0 * d * a.r));
			pto vec = (b.c - a.c) * (a.r / d);
			p.push_back((a.c + vec.rotar(alfa)).x);
			p.push_back((a.c + vec.rotar(-alfa)).x);
		}
	}
	sort(all(p));
	real res = 0.0;
	forn(i,p.size()-1)
	{
		const real A = p[i], B = p[i+1];
		VE ve; ve.reserve(2 * v.size());
		forn(j,v.size())
		{
			const circ &c = v[j];
			real arco = primitiva(B-c.c.x,c.r) - primitiva(A-c.c.x,c.r);
			real base = c.c.y * (B-A);
			ve.push_back(event(base + arco,-1));
			ve.push_back(event(base - arco, 1));
		}
		res += cuenta(ve,A,B);
	}
	return res;
}
\end{code}
\subsection{Integrador numerico (simpson).}
\begin{code}
typedef double Funcion(double);
double integrar(Funcion *f, double a,double b, int n)
{
	double h = (b-a)/(double)(n);
	double res = 0.0;
	double x0 = a;
	double fx0 = f(x0);
	const double h2 = h/2.0;
	forn(i,n)
	{
		double fx0h = f(x0+h);
		res += fx0 + fx0h + 4.0 * f(x0+h2);
		x0 += h;
		fx0 = fx0h;
	}
	return res * h / 6.0;
}
\end{code}
\subsection{Componentes biconexas, puentes y puntos de articulacion by Juancito}
\begin{code}
// g es la lista de adyacencia de un grafo en forma vector<int>, N es cantidad de vertices,
// MAXN es una cota superior tanto para cantidad de vertices como cantidad de aristas

int D[MAXN], L[MAXN], J[MAXN], I[MAXN]; char E[MAXN];
int P[2 * MAXN], R, S[2 * MAXN], K, A[MAXN], T;
void component() {
    int r = P[--R]; COMPO_START(); COMPO_V(r);
    for (int u = P[R - 1]; u != r; u = P[--R - 1]) {
        COMPO_V(u);
        if (D[P[R - 2]] < D[u]) forn(i, J[u]) COMPO_EDGE(u, g[u][i]);
    } COMPO_END();
}
void dfs (int r) {
    E[r] = K = T = R = 0; A[S[K++] = r] = -1;
    while (K) { int u = S[--K], v;
        switch (E[u]) {
        case 0: L[u] = D[u] = T++; J[u] = I[u] = 0; P[R++] = u;
        case 1: c1: if (I[u] == (int)g[u].size()) break;
            if (D[v = g[u][I[u]]] == -1) {
                //if(u == r and I[u]) ARTICULATION(u);
                E[A[v] = S[K++] = u] = 2, E[S[K++] = v] = 0;
            } else { if (v != A[u] && D[v] < L[u]) L[u] = D[v];
                if (D[v] < D[u]) swap(g[u][J[u]++], g[u][I[u]]); //COMP
                I[u]++; E[S[K++] = u] = 1;
            } break;
        case 2: v = g[u][I[u]], P[R++] = u;
            if (L[v] < L[u]) L[u] = L[v];
            //if (L[v] >= D[u] && u != r) ARTICULATION(u);
            //if (L[v] >= D[v]) BRIDGE(u, v);
            if (L[v] >= D[u]) component(); //COMP
            I[u]++; goto c1;
        }
    }
}
void BC()
{
    forn(i, N) D[i] = -1;
    forn(i, N) if(D[i] == -1) dfs(i);
}
\end{code}
\subsection{Rotaciones}
Matriz de rotacion 2D:
$$ \left[
  \begin{array}{ c c }
     \cos \theta & -\sin \theta \\
     \sin \theta & \cos \theta
  \end{array} \right]
$$

Matrices de rotacion 3D (sobre los ejes coordenados) :
$$R_x(\theta) = \left[ 
        \begin{array}{c c c}
            1 & 0 & 0 \\ 
            0 & \cos \theta & -\sin \theta \\
            0 & \sin \theta & \cos \theta \\
        \end{array} \right] $$ 
        
$$R_y(\theta) = \left[ 
        \begin{array}{c c c}
            \cos \theta & 0 & \sin \theta \\
            0 & 1 & 0 \\
            -\sin \theta & 0 & \cos \theta \\
        \end{array} \right] $$ 
        
$$R_z(\theta) = \left[ 
        \begin{array}{c c c}
            \cos \theta & -\sin \theta & 0 \\
            \sin \theta & \cos \theta & 0 \\
            0 & 0 & 1 \\
        \end{array} \right]$$

Rodrigues rotation formula (rota $\mathbf{v}$ alrededor de $\mathbf{z}$ vector unitario, segun un angulo $\theta$:
$$\mathbf{v}_\mathrm{rot} = \mathbf{v} \cos\theta + (\mathbf{z} \times \mathbf{v})\sin\theta   + \mathbf{z} (\mathbf{z} \cdot \mathbf{v}) (1 - \cos\theta)$$
\subsection{LIS}
\begin{code}
// Las lineas marcadas con // Camino no son necesarias si no se desea reconstruir el camino.

#define MAXN 1000000

int v[MAXN]; // INPU del algoritmo.
int mv[MAXN];
int mi[MAXN] ,p[MAXN]; // Camino
int l[MAXN]; // Aca apareceria la maxima subsecuencia creciente

int lis(int n)
{
	forn(i,n) mv[i] = INF;
	forn(i,n) mi[i] = -1; // Camino
	forn(i,n) p [i] = -1; // Camino
	mv[0] = -INF;
	int res = 0;
	forn(i,n)
	{
		// Con upper_bound es m�xima subsecuencia no decreciente.
		// Con lower_bound es m�xima subsecuencia creciente.
		int me = upper_bound(mv,mv+n,v[i]) - mv;
		p[i] = mi[me-1]; // Camino
		mv[me] = v[i];
		mi[me] = i; // Camino
		if (me > res) res = me;
	}
	for(int a = mi[res], i = res - 1;a != -1; a = p[a], i--) // Camino
		l[i] = a; // Indices: poniendo l[i] = v[a] quedan los valores.
	return res;
}
\end{code}
\subsection{Flujo de costo minimo vale multiejes}
\begin{code}
// Flujo de costo minimo, con lista de incidencia y flujo,cap,costo en los ejes.
// Se asumen costos y capacidades no negativos.
// Se banca ejes para los dos lados entre un par de nodos.
// SE BANCA MULTIEJES.
// O(m * n * F), siendo F el flujo que se pasa por la red.
#define MAXN 100
#define MAXM 10000
int S,T,N,M;
Cost co[MAXM];
Cap ca[MAXM], f[MAXM];
int g1[MAXM], g2[MAXM];

void iniG(int n,int s,int t) { N = n; S = s; T = t; M = 0; }
void aEje(int d,int h,Cap cap, Cost cost) {
	f[M] = 0;
	ca[M] = cap;
	co[M] = cost;
	g1[M] = d;
	g2[M] = h;
	M++;
}

const Cost INF = 1000000000000000000LL;
int p[MAXN];
Cost dist[MAXN];
inline void foo(int d,int h, Cost cost, int j, Cap mf) {
	if (mf > 0)	{
		Cost c = dist[d] + cost;
		if (c < dist[h]) { dist[h] = c; p[h] = j; }
	}
}

// camAu construye un camino aumentante de flujo a lo mas x, y pasa flujo por ahi.
// Al finalizar la ejecucion, x se ve reducido en la cantidad de flujo que se aumento.
// Devuelve el costo del camino en cuestion.
// Devuelve 0 si no se envia flujo (logico)
Cost camAu(Cap &x) {
	// Bellman ford.
	forn(i,N) {dist[i] =  INF; p[i] = -1;}
	dist[S] = 0;
	forn(i,N) forn(j,M) {
		int d = g1[j], h = g2[j];
		foo(d,h,co[j],j, ca[j] - f[j]);
		foo(h,d,-co[j],j, f[j]);
	} // aca ya tenemos computado el camino optimo para aumentar, si hay.
	int ac = T;
    Cap mF = x;
	while (p[ac] != -1)	{
		int j = p[ac];
		if (g1[j] == ac) { ac = g2[j]; mF = min(mF,f[j]);}
		else             { ac = g1[j]; mF = min(mF,ca[j] - f[j]); }
	}
	if (ac != S) return 0; // No hay camino.
	ac = T;
	while (p[ac] != -1) {
		int j = p[ac];
		if (g1[j] == ac) { ac = g2[j]; f[j] -= mF; }
		else             { ac = g1[j]; f[j] += mF; }
	}
	x -= mF;
	return mF * dist[T];
}

// flujo recibe la cantidad de flujo deseada (+inf para usar el flujo maximo).
// al finalizar la ejecucion, f queda con la cantidad de flujo pasada (que sera el valor pedido de ser posible,
// o bien el maximo flujo en la red sino).
// Devuelve el costo del flujo en cuestion.
Cost flujo(Cap &f) {
	Cap f0 = f, lf = f;
    Cost res = 0;
	while (1) {
		res += camAu(f);
		if (f == lf || f == 0) break;
		lf = f;
	}
	f = f0 - f;
	return res;
}
\end{code}
\subsection{Dual simple (dual sobre cada componente conexa)}
\begin{code}
// Usa : pto (resta), Orden total de puntos alrededor de un centro.
#define MAXN 1100
#define MAXM 6500

int m; // Cantidad de ejes
pto nodos[MAXN]; // nodos[i] es la coordenada del nodo i. INPU
int ea[2*MAXM],eb[2*MAXM]; // Cada eje va de ea[i] a eb[i]

void ginit() { m = 0; } // Funciones de entrada
void aEje(int a,int b) { ea[m] = a; eb[m++] = b;
                         ea[m] = b; eb[m++] = a;}
                         
int indi[2*MAXM]; // Indice del eje en la lista de adyacencia de a (nodo de partida)
vint g[MAXN]; // g : listas de adyacencia (dan los EJES que inciden en cada nodo).
int compo[MAXN]; // Componente conexa de cada vertice.

int reg[2*MAXM]; // Region que toca cada arista OUPU
vint dejes[MAXM]; // Lista de ejes de una region (caminando con mano izquierda en la pared) OUPU

Cmp micompa = Cmp(pto());
bool micmp(int e1, int e2) { return micompa(nodos[eb[e1]], nodos[eb[e2]]);}
bool mimen(pto a, pto b) { if (a.y != b.y) return a.y < b.y; return a.x < b.x; }
int tReg;
void workReg(int i) {
    dejes[tReg].clear();
    int ej = i;
    do {dejes[tReg].push_back(ej);
        reg[ej] = tReg;
        ej = g[eb[ej]][(indi[ej^1]+1) % g[eb[ej]].size()]; }
    while (ej != i);
    tReg++;
}
// Le pasas la cantidad de nodos y una variable donde te deja cant de comp. conexas
// Devuelve la cantidad de vertices del dual (regiones en el grafo original).
// En el grafo construido, las regiones 0..c-1 son las regiones exteriores de cada componente conexa.
// Nota que en realidad construye varios duales, uno por cada componente conexa.
// Si es conexo da el dual del grafo, sino da un grafo con tatas componentes como el original, y tal que
// cada componente es el dual de la correspondiente.
int buildDual(int n,int &c) {
    // Prepara el embedding para la otra parte
    forn(i,n) g[i].clear();
    forn(i,m) g[ea[i]].push_back(i);
    forn(i,n){
        micompa.r = nodos[i];
        sort(all(g[i]),micmp);
    }
    forn(i,n) forn(j,g[i].size()) indi[g[i][j]] = j;
    forn(i,m) reg[i] = -1;
    // Encuentra la region externa de cada dual
    forn(i,n) compo[i] = -1;
    tReg = 0; c = 0;
    forn(i,n)
    if (compo[i] == -1) {
        int menor = i;
        compo[i] = c;
        queue<int> q;
        q.push(i);
        while (!q.empty()){
            int x = q.front(); q.pop();
            if (mimen(nodos[x], nodos[menor])) menor = x;
            forall(ej,g[x]) {
                int y = eb[*ej];
                if (compo[y] == -1) {
                    compo[y] = c;
                    q.push(y);
        } } }
        c++;
        if (g[menor].empty()) dejes[tReg++].clear();
        else workReg(*g[menor].begin());
    }
    // Encuentra las demas regiones
    forn(i,m) if (reg[i] == -1) workReg(i);
    return tReg;
}
// EJEMPLO DE RECORRIDO DE LAS ARISTAS EN EL DUAL
// forn(i,tReg) forall(ej,dejes[i]) ARISTA_EN_EL_DUAL(i,reg[*ej^1]);
\end{code}
\subsection{Dual full}
\begin{code}
// usa: point in poly, area de un poligono
VP regPol[MAXM]; // Poligono que define el borde EXTERNO de una region (indefinido en la cara externa final)
                 // (dejes va a contener tambien los ejes en la frontera "interior")
// Luego de llamar a buildDual, llamas a este con el c que te dieron
// Te devuelve un N: El nuevo dual tiene vertices [c,N). [0,c) son basura.
// La unica cara externa tiene numero N-1
int buildFullDual(int c) {
    forn(i,tReg) { // Construccion de los poligonos
        regPol[i] = VP(dejes[i].size());
        forn(j,dejes[i].size())
            regPol[i][j] = nodos[ea[dejes[i][j]]];
    } // Ahora si el full dual
    dejes[tReg].clear();
    forn(i,c)
    if (!dejes[i].empty()) {
        int mejor = tReg;
        tint mar = AREA_INF;
        pto p = nodos[ea[dejes[i][0]]];
        forsn(j,c,tReg)
        if (compo[ea[dejes[j][0]]] != i) {
            if (pnpoly(regPol[j],p)) {
                tint ar = areaPor2(regPol[j]);
                if (ar < mar) {
                    mejor = j;
                    mar = ar;
        } } }
        // Enchufar la region i a la mejor
        forall(ej,dejes[i]) {
            reg[*ej] = mejor;
            dejes[mejor].push_back(*ej);
        }
    }
    return ++tReg;
}
\end{code}
\subsection{LCA}
\begin{code}
// La cantidad maxima de vertices n debe ser menor que 2^(LVL-1).
#define LVL 18
#define MAXN (1<<LVL)
int d[MAXN];

bool fcmp(int a,int b) { return d[a] < d[b];}

int vec[LVL][MAXN];
int mn(int i, int j) { // intervalo [i,j)
	int p = 31-__builtin_clz(j-i);
	return min(vec[p][i],vec[p][j-(1<<p)],fcmp);
}
void mn_init(int n) {
	int mp = 31-__builtin_clz(n);
	forn(p, mp) forn(x, n-(1<<p)) vec[p+1][x] = min(vec[p][x], vec[p][x+(1<<p)],fcmp);
}

vector<int> t[MAXN]; // Tree
int pos[MAXN];
int id, pp;

void dfs(int x) {
	d[x] = id++;
	vec[0][pos[x] = pp++] = x;
	forall(y,t[x])
	if (d[*y] == -1) {
		dfs(*y);
		vec[0][pp++] = x;
	}
}

void lcaInit(int n,int raiz) {
	id = pp = 0;
	memset(d,-1,sizeof(d));
	dfs(raiz);
	mn_init(2*n-1);
}

int lca(int i,int j) {
	int a = pos[i], b = pos[j];
	if (a > b) swap(a,b);
	return mn(a,b+1);
}
\end{code}
\subsection{Interseccion semiplano-poligono convexo O(n)}
\begin{code}
// Usa: pto (con doubles) (+ - , producto cruz ^, producto por un escalar *, ==)
const double EPS = 1e-9;
pto irs(pto a,pto b,pto p0, pto p1) {
    #define onr(p) (fabs((b-a)^(p-a)) < EPS)
    if (onr(p1)) return p1;
    if (onr(p0)) return p0;
    return p0 + (p1-p0) * (((a-p0)^(b-a)) / ((p1-p0)^(b-a)));
}
// Parado en a, mirando hacia b, interseca el semiplano de la izquierda con el poligono convexo p
// Un VP vacio representa el conjunto vacio.
VP cortar(const VP &p, pto a, pto b) {
    #define inside(p) (((b-a)^(p-a)) >= -EPS)
    int n = p.size();
    VP res; if (n==0) return p;
    int in = inside(p[n-1]);
    for (int i=0,j=n-1;i<n;j=i++) {
        int nin = inside(p[i]);
        if (nin != in) {
            in = nin;
            res.push_back(irs(a,b,p[j],p[i]));
        }
        if (in) res.push_back(p[i]);
    }
    res.resize(unique(all(res)) - res.begin());
    while (res.size() > 1 && res.back() == res.front()) res.pop_back();
    return res;
}
\end{code}
\subsection{Distancia punto-triangulo en 3D}
\begin{code}
struct pto
{
    tipo x, y, z;
    pto() : x(0), y(0), z(0) {}
    pto(tipo x0, tipo y0, tipo z0) : x(x0), y(y0), z(z0) {}
    pto(const pto& p) : x(p.x), y(p.y), z(p.z) {}
    pto operator + (pto& p) { return pto(x + p.x, y + p.y, z + p.z); }
    pto operator - (pto& p) { return pto(x - p.x, y - p.y, z - p.z); }
    tipo operator * (pto& p) { return x * p.x + y * p.y + z * p.z; }
    pto operator * (tipo a) { return pto(x * a, y * a, z * a); }
    tipo norma2() { return x * x + y * y + z * z; }
    tipo dis2(pto& p) { return sqr(x - p.x) + sqr(y - p.y) + sqr(z - p.z); }
};
inline tipo dis2(pto p1, pto p2){ return p1.dis2(p2); }

/**
 * tengo a, b, c y quiero proyectar c en a + (b - a) * t
 * resto a a todo:
 * tengo b - a, c - a y quiero proyectar c - a en (b - a) * t
 * es (b - a) * ((c - a) * (b - a)) / ((b - a) * (b - a))
 * es la misma cuenta de antes
 */

tipo dis2puntosegmento(pto a, pto b, pto c)
{
    pto ba = b - a, ca = c - a, bc = b - c;
    tipo t = (ca * ba) / (ba * ba);
    if(0 <= t and t <= 1) {
        pto proy = ba * t;
        pto normal = ca - proy;
        return normal.norma2();
    }
    else return min(dis2(a, c), dis2(b, c));
}

tipo dis2rectarecta(pto a, pto b, pto c, pto d)
{
    tipo res = dis2puntosegmento(a, b, c);
    tipo a11 = ba * ba, a12 = -(dc * ba), a21 = ba * dc, a22 = -(dc * dc);
    tipo det = a11 * a22 - a12 * a21;
    if(zero(det)) return res;
    swap(a11, a22); a12 = -a12; a21 = -a21;
    tipo t1 = (a11 * (ca * ba) + a12 * (ca * dc)) / det;
    tipo t2 = (a21 * (ca * ba) + a22 * (ca * dc)) / det;
    ba = ba * t1, dc = dc * t2;
    return dis2(ba, ca + dc);
}

tipo dis2segseg(pto a, pto b, pto c, pto d)
{
    tipo res = INF;
    res = min(res, dis2puntosegmento(a, b, c));
    res = min(res, dis2puntosegmento(a, b, d));
    res = min(res, dis2puntosegmento(c, d, a));
    res = min(res, dis2puntosegmento(c, d, b));
    
    pto ba = b - a, dc = d - c, ca = c - a;
    tipo a11 = ba * ba, a12 = -(dc * ba), a21 = ba * dc, a22 = -(dc * dc);
    tipo det = a11 * a22 - a12 * a21;
    if(zero(det)) return res;
    else
    {
        swap(a11, a22); a12 = -a12; a21 = -a21;
        tipo t1 = (a11 * (ca * ba) + a12 * (ca * dc)) / det;
        tipo t2 = (a21 * (ca * ba) + a22 * (ca * dc)) / det;
        if(0 <= t1 and t1 <= 1 and 0 <= t2 and t2 <= 1)
        {
            ba = ba * t1, dc = dc * t2;
            return min(res, dis2(ba, ca + dc));
        }
        else return res;
    }
}

/**
 * tengo un tri�ngulo a, b, c y un punto x
 * quiero ver la distancia m�nima de x en (a, b, c)
 * si es la proyecci�n
 */

tipo mindis2puntotrialgulo(pto a, pto b, pto c, pto x)
{
    pto ba = b - a, ca = c - a, xa = x - a, caba = ca - ba;
    tipo a11 = ba * ba, a12 = ba * caba, a21 = ba * caba, a22 = caba * caba;
    tipo det = a11 * a22 - a12 * a21;
    if(zero(det)) return INF;
    else
    {
        swap(a11, a22); a12 = -a12; a21 = -a21;
        tipo t1 = (a11 * (xa * ba) + a12 * (xa * caba)) / det;
        tipo t2 = (a21 * (xa * ba) + a22 * (xa * caba)) / det;
        if(0 <= t2 and t2 <= t1 and t1 <= 1)
        {
            ba = ba * t1; caba = caba * t2;
            return dis2(xa, ba + caba);
        }
        else return INF;
    }
}
\end{code}
\subsection{Algoritmo de Duval}
    Dada una string $s$ devuelve la Lyndon decomposition en tiempo lineal usando el algoritmo de Duval.
    Factoriza $s$ como $s_1 s_2 \ldots s_k$ con $s_1 \geqq s_2 \geqq \cdots \geqq s_k$ y tal que $s_i$ es Lyndon, esto es, es su menor rotaci�n.
\begin{code}
void duval(char* s) {   
    int i = 0, n = strlen(s), j, k;
    while (i < n) {
        j = i + 1, k = i;
        while (j < n and s[k] <= s[j]) {
            if(s[k] < s[j]) k = i; else k++; j++; }
        while (i <= k) {
        		LYNDON(i, i + j - k); i += j - k; }}}
\end{code}
\noindent Obtener la m�nima rotaci�n de $s$: en la descomposici�n de Lyndon de $s^2$ es el �ltimo $i<|s|$ con el que empieza una Lyndon.\\
\noindent Dada una string s devuelve un array m[0:n] tal que m[i] contine el m�nimo sufijo de s[0:i+1].
\begin{code}
void minimumSuffixArray (char* s, int* res) {
    int i = 0, n = strlen(s), j, k;
    while (i < n) {
        j = i + 1; k = i; res[i] = i;
        while (j < n and s[k] <= s[j]) {
            if (s[k] < s[j]) res[j] = k = i;
            else res[j] = j - k + res[k], k++; j++; }
        while (i <= k) i += j - k; }}
\end{code}
\noindent Dada una string s devuelve un array m[0:n] tal que m[i] contine el m�ximo sufijo de s[0:i+1].
\begin{code}
void maximumSuffixArray (char* s, int* res) {
    int i = 0, n = strlen(s), j, k; forn(l, n) res[l] = -1;
    while (i < n) {
        j = i + 1; k = i;
        if (res[i] == -1) res[i] = i;
        while (j < n and s[k] >= s[j]) {
            if (s[k] > s[j]) k = i; else k++;
            if (res[j] == -1) res[j] = i; j++; }
        while (i <= k) i += j - k; }}
\end{code}
\end{document}
